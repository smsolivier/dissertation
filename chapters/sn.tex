%!TEX root = ../doc.tex
\documentclass[../doc.tex]{subfiles}

\begin{document}
\chapter{Transport Discretization}
This chapter presents the \gls{sn} and \gls{dg} discretization for the Boltzmann transport equation. We consider the steady-state, mono-energetic transport problem with isotropic scattering given by: 
	\begin{subequations}
	\begin{equation}
		\Omegahat\cdot\nabla\psi + \sigma_t \psi = \frac{\sigma_s}{4\pi}\int \psi \ud \Omega' + q \,, \quad \x \in \D \,,
	\end{equation}
	\begin{equation}
		\psi(\x,\Omegahat) = \bar{\psi}(\x,\Omegahat) \,, \quad \x \in \partial\D \ \mathrm{and} \ \Omegahat\cdot\n < 0 \,, 
	\end{equation}
	\end{subequations}
where $\psi(\x,\Omegahat)$ is the angular flux, $\D$ the domain of the problem with $\partial\D$ its boundary, $\sigma_t(\x)$ and $\sigma_s(\x)$ the total and scattering macroscopic cross sections, respectively, $q(\x,\Omegahat)$ the fixed-source, and $\bar{\psi}(\x,\Omegahat)$ the inflow boundary function. 

\section{Discrete Ordinates}
The \gls{sn} angular model collocates the transport equation at a set of discrete angles, $\Omegahat_d$, and integration is numerically approximated using a suitable angular quadrature rule $\{\Omegahat_d,w_d\}_{d=1}^{N_\Omega}$ on the unit sphere. The discrete-in-angle transport equation 
	\begin{subequations}
	\begin{equation}
		\Omegahat_d \cdot\nabla\psi_d + \sigma_t\psi_d = \frac{\sigma_s}{4\pi} \sum_{d'=1}^{N_\Omega} w_{d'} \psi_{d'} + q_d \,, \quad \x \in \D \,,
	\end{equation}
	\begin{equation}
		\psi_d(\x) = \bar{\psi}(\x,\Omegahat_d) \,, \quad \x \in \partial\D \ \mathrm{and} \ \Omegahat_d\cdot\n < 0 \,,
	\end{equation}
	\end{subequations}
where $d \in [1,N_\Omega]$, $\psi_d(\x) = \psi(\x,\Omegahat_d)$ and $q_d(\x) = q(\x,\Omegahat_d)$ are the angular flux and fixed-source of particles traveling in the discrete direction $\Omegahat_d$, respectively. Figure \ref{sn:level_sym} shows an example of an octant of the Level Symmetric $S_6$ angular quadrature rule. A key property of \gls{sn} is that the angles are only coupled in the scattering term. That is, if the scattering source were known each $\psi_d(\x)$ could be solved for independently. 
% --- level symmetric octant --- 
\begin{figure}
\centering
\includegraphics[width=.65\textwidth]{figs/level_sym.pdf}
\caption{The positive octant for the level symmetric $S_6$ angular quadrature rule. }
\label{sn:level_sym}
\end{figure}

\section{Discontinuous Galerkin}
We now apply a \gls{dg} discretization to the \gls{sn} transport equations. We derive a discretization for each angle independently by approximating each $\psi_d$ in the degree-$p$ DG space $Y_p$ introduced in Section \ref{fem_sec:dg}. The weak form is first derived on each element $K$. A global approximation is found by defining the upwind numerical flux that couples adjacent elements based on the direction of $\Omegahat_d$. We delay discussion of the computation of the scattering source until Section \ref{sn_sec:vef} and assume for the moment that the scattering source is included in the fixed-source $q$. 

% --- upwind diagram --- 
\begin{figure}
\centering
\includegraphics[width=.65\textwidth]{figs/upwind.pdf}
\caption{A depiction of a grouping of mesh elements where, due to the direction of $\Omegahat$, the element $K_1$ is upwind of $K_2$.}
\label{sn:upwind_diag}
\end{figure}
The weak form on each element is: find $\psi_d \in Y_p$ such that for all $K \in \meshT$: 
	\begin{equation}
		\int_{\partial K} \Omegahat_d \cdot\n\, u \widehat{\psi}_d \ud s - \int_K \Omegahat_d\cdot\nabla u \, \psi_d \ud \x + \int_K \sigma_t\, u\psi_d \ud \x = \int_K u\, q_d \ud \x \,, 
	\end{equation}
where the numerical flux $\widehat{\psi}_d$ is either an approximation of $\psi_d$ on interior mesh interfaces or given by the inflow boundary function $\bar{\psi}$ on an inflow boundary. We use the upwind numerical flux that defines the incoming angular flux as the outflow from the upwind element. On a face $\mathcal{F}$ between elements $K_1$ and $K_2$ with normal pointing from $K_1$ to $K_2$ (see Fig.~\ref{sn:upwind_diag}), the upwind numerical flux is defined as 
	\begin{equation} \label{sn:upwind}
		\widehat{\psi}_d = \begin{cases}
			\psi_{d,1} \,, & \Omegahat_d\cdot\n > 0 \\
			\psi_{d,2} \,, & \Omegahat_d\cdot\n < 0 
		\end{cases} \,, \quad \forall \mathcal{F} \in \Gamma_0 \,, 
	\end{equation}
where $\psi_{d,i} = \psi_d|_{K_i}$. For boundary faces, we set 
	\begin{equation} \label{sn:upwind_bdr}
		\widehat{\psi}_d(\x) = \begin{cases}
			\bar{\psi}(\x,\Omegahat_d) \,, & \Omegahat\cdot\n < 0 \\
			\psi_d(\x) \,, & \Omegahat\cdot\n > 0 
		\end{cases} \,, \quad \forall \mathcal{F} \in \Gamma_b \,. 
	\end{equation}
Note that for $\mathcal{F} \in \Gamma_b$, we use the convention that $\n$ is the outward unit normal. Thus, Eq.~\ref{sn:upwind_bdr} applies the inflow boundary condition when $\Omegahat_d\cdot\n<0$. Observe that the conditions in Eqs.~\ref{sn:upwind} and \ref{sn:upwind_bdr} can equivalently be written using the switch functions: 
	\begin{equation}
		\Omegahat\cdot\n\,\widehat{\psi}_d = \begin{cases}
			\frac{1}{2}\paren{\Omegahat\cdot\n + |\Omegahat\cdot\n|}\!\psi_{d,1} + \frac{1}{2}(\Omegahat\cdot\n - |\Omegahat\cdot\n|)\psi_{d,2} \,, & \mathcal{F} \in \Gamma_0 \\ 
			\frac{1}{2}(\Omegahat\cdot\n + |\Omegahat\cdot\n|)\psi_{d} + \frac{1}{2}(\Omegahat\cdot\n - |\Omegahat\cdot\n|)\bar{\psi}(\x,\Omegahat_d) \,, & \mathcal{F} \in \Gamma_b 
		\end{cases} \,, 
	\end{equation}
since when $\Omegahat\cdot\n>0$, $\frac{1}{2}(\Omegahat\cdot\n + |\Omegahat\cdot\n|) = \Omegahat\cdot\n$ and $\frac{1}{2}(\Omegahat\cdot\n - |\Omegahat\cdot\n|) = 0$ with the opposite holding when $\Omegahat\cdot\n<0$. Using the definitions of the jump and average in Eq.~\ref{fem:jump_avg}, the switch functions can be rewritten as 
	\begin{equation}
		\Omegahat\cdot\n\,\widehat{\psi}_d = 
			\Omegahat\cdot\n\avg{\psi_d} + \frac{1}{2}|\Omegahat\cdot\n|\jump{\psi_d} \,, \quad \mathcal{F} \in \Gamma_0 
	\end{equation}
with the boundary case left unchanged. Note that $\widehat{\psi}_d$ is single-valued on all faces in the mesh. Thus, the jumps and averages identity (Eq.~\ref{fem:jumps_avg_id}) simplifies to 
	\begin{equation}
		\jump{u\widehat{\psi}_d} = \jump{u} \widehat{\psi}_d \,. 
	\end{equation}

Using the upwind numerical flux and summing over all elements yields the global weak form: find $\psi_d \in Y_p$ such that 
	\begin{multline}
		\frac{1}{2}\int_{\Gamma_b} u\!\paren{\Omegahat_d\cdot\n + |\Omegahat_d\cdot\n|}\psi_d \ud s + \int_{\Gamma_0} \jump{u}\!\paren{\Omegahat_d\cdot\n\avg{\psi_d} + \frac{1}{2}|\Omegahat_d\cdot\n|\jump{\psi_d}} \ud s - \int \Omegahat_d\cdot\nablah u \, \psi_d \ud \x \\+ \int \sigma_t\, u \psi_d \ud \x = \int u\,q_d \ud \x - \frac{1}{2}\int_{\Gamma_b} u\!\paren{\Omegahat_d\cdot\n - |\Omegahat_d\cdot\n|}\bar{\psi}(\x,\Omegahat_d) \ud s \,, \quad \forall u \in Y_p \,,
	\end{multline}
where $\nablah$ denotes the broken gradient defined in Eq.~\ref{fem:broken_grad}. Defining the bilinear forms 
	\begin{subequations}
	\begin{equation}
		\underline{u}^T\mat{M}_t \underline{\psi}_d = \int \sigma_t\, u\psi_d \ud \x \,,
	\end{equation}
	\begin{equation}
		\underline{u}^T \mat{G} \underline{\psi}_d = -\int \Omegahat\cdot\nabla u \, \psi_d \ud \x \,,
	\end{equation}
	\begin{equation}
		\underline{u}^T \mat{F} \underline{\psi}_d = \frac{1}{2}\int_{\Gamma_b} u\!\paren{\Omegahat_d\cdot\n + |\Omegahat_d\cdot\n|}\psi_d \ud s + \int_{\Gamma_0} \jump{u}\!\paren{\Omegahat_d\cdot\n\avg{\psi_d} + \frac{1}{2}|\Omegahat_d\cdot\n|\jump{\psi_d}} \ud s \,,
	\end{equation}
	\end{subequations}
and the linear form 
	\begin{equation}
		\underline{u}^T \underline{b} = \int u\,q_d \ud \x - \frac{1}{2}\int_{\Gamma_b} u\!\paren{\Omegahat_d\cdot\n - |\Omegahat_d\cdot\n|}\bar{\psi}(\x,\Omegahat_d) \ud s \,, 
	\end{equation}
the discrete transport system is: 
	\begin{equation}
		\paren{\mat{F} + \mat{G} + \mat{M}_t}\underline{\psi}_d = \underline{b} \,, \quad 1 \leq d \leq N_{\Omega} \,. 
	\end{equation}
Since the space $Y_p$ does not share degrees of freedom across interior element interfaces, the matrices $\mat{G}$ and $\mat{M}_t$ are block diagonal by element. However, due to the coupling between elements in the numerical flux, $\mat{F}$ has additional couplings. 

% --- reentrant diagram --- 
\begin{figure}
\centering
\includegraphics[width=.65\textwidth]{figs/reentrant.pdf}
\caption{A mesh of four cubic elements. The face $\mathcal{F} = K_1 \cap K_3$ is curved such that $K_3$ has a concave face. For the direction $\Omegahat$, this means particles can originate in $K_3$, exit into $K_1$, and then reenter $K_3$. Note that $\mathcal{F} = K_2 \cap K_4$ is also reentrant in the direction $\Omegahat$.}
\label{sn:reentrant_diag}
\end{figure}

% --- companion to reentrant diagram ---
\begin{figure}
\foreach \f in {1,...,4}{
	\begin{subfigure}{.49\textwidth}
		\centering
		\includegraphics[width=\textwidth]{{figs/reentrant_\f.pdf}}
		\caption{}
	\end{subfigure}
}
\caption{(a) a plot of $K_3$ from Fig.~\ref{sn:reentrant_diag} showing the normal vectors along the concave face along the bottom of $K_3$. For the depicted direction $\Omegahat$, $\Omegahat\cdot\n$ changes sign along the face as shown by the changing colors of the normal vectors. (b) plots $\Omegahat\cdot\n$ as a function of the reference coordinate, $\xi \in [0,1]$, corresponding to the bottom face of $K_3$. (c) and (d) plot the upwind and downwind parts of $\Omegahat\cdot\n$, respectively. Observe that the upwind and downwind parts have discontinuous derivatives.}
\label{sn:reentrant_plots}
\end{figure}

\section{The Transport Sweep}
% --- sweep plot --- 
\begin{figure}
\centering
\begin{subfigure}{.49\textwidth}
	\centering
	\includegraphics[width=\textwidth]{figs/sweep.pdf}
	\caption{}
\end{subfigure}
\begin{subfigure}{.49\textwidth}
	\centering
	\includegraphics[width=\textwidth]{figs/sweep_1.pdf}
	\caption{}
\end{subfigure}
\caption{Sparsity plots for a 1D DG \gls{sn} transport problem. The unknowns stride in space and then angle. In (a), scattering is included showing that all angles are coupled. By lagging the scattering term, (b) shows that each angle can now be solved independently. Furthermore, for each angle the system is either lower or upper block triangular by element. This means each angle can be solved element-by-element in a ``sweep''. In two and three dimensions, the block triangular structure can be revealed by a suitable reordering of the elements.}
\label{sn:sweep}
\end{figure}

\section{Discrete VEF Data}

\section{Positivity Preserving Flux Fixups}

\section{Connection to VEF Algorithm} \label{sn_sec:vef}

\end{document}