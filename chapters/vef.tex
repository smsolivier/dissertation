%!TEX root = ../doc.tex
\documentclass[../doc.tex]{subfiles}

\begin{document}
\chapter{The VEF Algorithm}

\section{Linear Transport Equation}
The steady-state, mono-energetic, fixed-source transport problem with isotropic scattering and inflow boundary conditions is: 
	\begin{subequations}
	\begin{equation} \label{eq:transport}
		\Omegahat\cdot\nabla\psi + \sigma_t \psi = \frac{\sigma_s}{4\pi}\int \psi \ud \Omega' + q \,, \quad \x \in \D \,,
	\end{equation}
	\begin{equation}
		\psi(\x,\Omegahat) = \bar{\psi}(\x,\Omegahat) \,, \quad \x \in \partial \D \ \mathrm{and} \ \Omegahat\cdot\n < 0 \,, 
	\end{equation}
	\end{subequations}
where $\psi(\x,\Omegahat)$ is the angular flux, $\D$ the domain of the problem with $\partial\D$ its boundary, $\sigma_t(\x)$ and $\sigma_s(\x)$ the total and scattering macroscopic cross sections, respectively, $q(\x,\Omegahat)$ the fixed-source, and $\bar{\psi}(\x,\Omegahat)$ the inflow boundary function. 
The independent variables are $\x \in \R^{\dim}$ and $\Omegahat \in \mathbb{S}^2$ where $\x$ defines a location in $\dim$-dimensional space and $\Omegahat$ a direction on the unit sphere, $\mathbb{S}^2$. The direction variable is represented using the polar and azimuthal angles $(\theta,\varphi)$ as shown in Fig. \ref{vef:omega_diagram}. Its projections onto the Cartesian coordinate axes $(\e_x, \e_y, \e_z)$ are given by: 
	\begin{equation}
		\mu = \sin \varphi \cos\theta \,, \quad \eta = \sin\varphi \sin\theta \,, \quad \xi = \cos\varphi \,,  
	\end{equation}
so that $\Omegahat = \mu\e_x + \eta\e_y + \xi\e_z$. The streaming term can then be written as 
	\begin{equation}
		\Omegahat\cdot\nabla\psi = \mu\pderiv{\psi}{x} + \eta\pderiv{\psi}{y} + \xi\pderiv{\psi}{z} \,. 
	\end{equation}

% --- spherical coordinates for Omega --- 
\begin{figure}
\centering
\includegraphics[width=.65\textwidth]{figs/omega.pdf}
\caption{A depiction of the spherical coordinate system used for the direction of particle travel variable, $\Omegahat$. Here, $\theta \in [0,2\pi]$ is the azimuthal angle and $\varphi \in [0,\pi]$ the polar angle. }
\label{vef:omega_diagram}
\end{figure}

% --- VEF commuting diagram --- 
\begin{figure}
\centering
\includegraphics[width=.85\textwidth]{figs/vef_commuting.pdf}
\caption{The commuting diagram for VEF approximations of the transport equation. Consistent VEF methods apply discrete closures to a discretized transport equation whereas the independent methods discretize the continuous VEF equations formed through continuous closures of the continuous transport equation. In both cases, the discrete VEF system is an approximation to the continuous transport equation. However, independent methods generally do not have equivalence of the bottom row. That is discrete VEF and transport are in general not equivalent.}
\label{vef:commuting}
\end{figure}

\end{document}