%!TEX root = ../doc.tex
\documentclass[../doc.tex]{subfiles}

\begin{document}
\chapter{The VEF Algorithm}

\section{Linear Transport Equation}
The steady-state, mono-energetic, fixed-source transport problem with isotropic scattering and inflow boundary conditions is: 
	\begin{subequations}
	\begin{equation} \label{eq:transport}
		\Omegahat\cdot\nabla\psi + \sigma_t \psi = \frac{\sigma_s}{4\pi}\int \psi \ud \Omega' + q \,, \quad \x \in \D \,,
	\end{equation}
	\begin{equation}
		\psi(\x,\Omegahat) = \bar{\psi}(\x,\Omegahat) \,, \quad \x \in \partial \D \ \mathrm{and} \ \Omegahat\cdot\n < 0 \,, 
	\end{equation}
	\end{subequations}
where $\psi(\x,\Omegahat)$ is the angular flux, $\D$ the domain of the problem with $\partial\D$ its boundary, $\sigma_t(\x)$ and $\sigma_s(\x)$ the total and scattering macroscopic cross sections, respectively, $q(\x,\Omegahat)$ the fixed-source, and $\bar{\psi}(\x,\Omegahat)$ the inflow boundary function. 
The independent variables are $\x \in \R^{\dim}$ and $\Omegahat \in \mathbb{S}^2$ where $\x$ defines a location in $\dim$-dimensional space and $\Omegahat$ a direction on the unit sphere, $\mathbb{S}^2$. The direction variable is represented using the polar and azimuthal angles $(\theta,\varphi)$ as shown in Fig. \ref{fig:omega_diagram}. Its projections onto the Cartesian coordinate axes $(\e_x, \e_y, \e_z)$ are given by: 
	\begin{equation}
		\mu = \sin \varphi \cos\theta \,, \quad \eta = \sin\varphi \sin\theta \,, \quad \xi = \cos\varphi \,,  
	\end{equation}
so that $\Omegahat = \mu\e_x + \eta\e_y + \xi\e_z$. The streaming term can then be written as 
	\begin{equation}
		\Omegahat\cdot\nabla\psi = \mu\pderiv{\psi}{x} + \eta\pderiv{\psi}{y} + \xi\pderiv{\psi}{z} \,. 
	\end{equation}

\begin{figure}
\centering
\includegraphics[width=.65\textwidth]{figs/omega.pdf}
\caption{}
\label{fig:omega_diagram}
\end{figure}

\end{document}