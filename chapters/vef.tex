%!TEX root = ../doc.tex
\documentclass[../doc.tex]{subfiles}

\begin{document}
\chapter{The VEF Algorithm}
In this document, we develop \gls{vef} methods for the linear, steady-state, mono-energetic, fixed-source transport problem with isotropic scatttering. This problem represents the basic kernel for many more complicated problems such as \gls{trt}. 
The linear Boltzmann equation that models this problem is: 
	\begin{subequations}
	\begin{equation} \label{vef:transport}
		\Omegahat\cdot\nabla\psi + \sigma_t \psi = \frac{\sigma_s}{4\pi}\int \psi \ud \Omega' + q \,, \quad \x \in \D \,,
	\end{equation}
	\begin{equation} \label{vef:inflow}
		\psi(\x,\Omegahat) = \bar{\psi}(\x,\Omegahat) \,, \quad \x \in \partial \D \ \mathrm{and} \ \Omegahat\cdot\n < 0 \,, 
	\end{equation}
	\end{subequations}
where $\psi(\x,\Omegahat)$ is the angular flux, $\D$ the domain of the problem with $\partial\D$ its boundary, $\sigma_t(\x)$ and $\sigma_s(\x)$ the total and scattering macroscopic cross sections, respectively, $q(\x,\Omegahat)$ the fixed-source, and $\bar{\psi}(\x,\Omegahat)$ the inflow boundary function. 

In this chapter, we derive the \gls{vef} algorithm by taking angular moments of the transport equation given by Eq.~\ref{vef:transport}. Due to the streaming term, $\Omegahat\cdot\nabla\psi$, angular moments always produce more unknowns than equations. In VEF, an exact, trivial closure of the second moment is used so that the moment system is an \emph{equivalent reformulation} of the transport problem. Since the VEF closures are weak functions of the solution, iterative scheme can be designed to converge rapidly. We also discuss the mathematical properties of the VEF data and their impact on the structure of the VEF equations. The section concludes with a discussion of the two primary philosophies used to design discrete VEF methods. 

\section{Derivation of the VEF Equations}
Integrating the transport equation over angle yields 
	\begin{equation}
		\nabla\cdot\vec{J} + \sigma_a \varphi = Q_0 \,, 
	\end{equation}
where $\varphi$ and $\vec{J}$ are the zeroth and first angular moments of the angular flux, $Q_0$ the zeroth moment of the fixed source, $q$, and $\sigma_a(\x) = \sigma_t(\x) - \sigma_s(\x)$ the absorption cross section. Multiplying by $\Omegahat$ and integrating over angle yields the first moment equation: 
	\begin{equation}
		\nabla\cdot\P + \sigma_t \vec{J} = \vec{Q}_1 \,, 
	\end{equation}
where $\P = \int \Omegahat\otimes\Omegahat\,\psi \ud \Omega$ is the second moment of $\psi$ and $\vec{Q}_1$ is the first angular moment of $q$. We refer to the first three moments as the scalar flux, current, and pressure, respectively. In three dimensions, we have 10 unknowns corresponding to the scalar flux, three components of the current, and the six unique components of the symmetric pressure tensor but only four equations arising from the scalar zeroth moment and vector first moment equations. 

VEF introduces an additional equation, known as a closure, that expresses the pressure in terms of the scalar flux. This is achieved with an exact but trivial multiplicative closure of the form: 
	\begin{equation}
		\P = \E\varphi \,,
	\end{equation}
where 
	\begin{equation}
		\E = \frac{\int \Omegahat\otimes\Omegahat\,\psi \ud \Omega}{\int \psi \ud \Omega} 
	\end{equation}
is the Eddington tensor. Note that this closure is an algebraic reformulation of the pressure derived by multiplying and dividing by the scalar flux. This is a trivial closure in that the solution to the transport equation, $\psi$, must be known to define the Eddington tensor. The VEF equations are then: 
	\begin{subequations} \label{vef:vefeq}
	\begin{equation}
		\nabla\cdot\vec{J} + \sigma_a \varphi = Q_0 \,, 
	\end{equation}
	\begin{equation}
		\nabla\cdot\paren{\E\varphi} + \sigma_t \vec{J} = \Qone \,. 
	\end{equation}
	\end{subequations}
By eliminating the current, the VEF equations can be cast as a drift-diffusion equation: 
	\begin{equation} \label{vef:drift}
		-\nabla\cdot\frac{1}{\sigma_t}\nabla\cdot\paren{\E\varphi} + \sigma_a \varphi = Q_0 - \nabla\cdot\frac{\Qone}{\sigma_t} \,. 
	\end{equation}
In both the first-order form (Eq.~\ref{vef:vefeq}) and the second-order form (Eq.~\ref{vef:drift}), the presence of the Eddington tensor inside the divergence leads to diffusion, advection, and reaction-like terms that make applying existing discretization techniques difficult. 

The Miften-Larsen boundary conditions \cite{QDBC} are derived using an analogous closure. Let $J_n^\pm = \int_{\Omegahat\cdot\n\gtrless 0} \Omegahat\cdot\n\, \psi \ud \Omega$ be the positive and negative partial currents such that $\vec{J}\cdot\n = J_n^+ + J_n^-$. We can then write, 
	\begin{equation}
	\begin{aligned}
		\vec{J}\cdot\n &= J_n^+ + J_n^- \\ 
		&= (J_n^+ - J_n^-) + 2J_n^- \\
		&= \int |\Omegahat\cdot\n|\, \psi \ud \Omega + 2J_n^- \,. 
	\end{aligned}
	\end{equation}
Here, it is assumed that the incoming partial current, $J_n^-$, is computed from the inflow boundary condition, $\bar{\psi}$, and that $\int |\Omegahat\cdot\n|\,\psi \ud \Omega$ is computed from the transport solution. Applying a similar multiplicative closure yields: 
	\begin{equation}
		\vec{J}\cdot\n = E_b \varphi + 2J_n^- \,,
	\end{equation}
where 
	\begin{equation}
		E_b = \frac{\int |\Omegahat\cdot\n|\,\psi \ud \Omega}{\int \psi \ud \Omega} 
	\end{equation}
is the Eddington boundary factor. The VEF moment system with boundary conditions is
	\begin{subequations}
	\begin{equation}
		\nabla\cdot\vec{J} + \sigma_a \varphi = Q_0 \,, \quad \x \in \D \,, 
	\end{equation}
	\begin{equation}
		\nabla\cdot\paren{\E\varphi} + \sigma_t\vec{J} = \Qone \,, \quad \x \in \D \,,
	\end{equation}
	\begin{equation} \label{vef:mlbc}
		\vec{J}\cdot\n = E_b \varphi + 2\Jin \,, \quad \x \in \partial \D \,,
	\end{equation}
	\end{subequations}
where $\Jin = \int_{\Omegahat\cdot\n<0} \Omegahat\cdot\n\, \bar{\psi} \ud \Omega$ is the incoming partial current computed using the inflow boundary function, $\bar{\psi}$. 

If the Eddington tensor and boundary factor are known, the VEF equations with Miften-Larsen boundary conditions define the zeroth and first moments of the angular flux. That is, they are an equivalent reformulation of the transport equation. However, this is a trivial closure in that the solution to the transport equation must be known in order to define the VEF data. VEF methods rely on the fact that the VEF data are weak functions of the solution and thus simple iterative schemes can converge rapidly. 

Note that when an independent discretization is used for the VEF equations, the discretized VEF scalar flux and VEF current will not be equivalent to the zeroth and first angular moments of the discrete angular flux; the two solutions will differ on the order of the spatial discretization error. To notationally separate the two scalar flux solutions, we use $\varphi$ (varphi) to denote the VEF scalar flux and $\phi = \int \psi \ud \Omega$ (phi) for the zeroth moment of the angular flux. 

\section{The VEF Algorithm}
VEF methods seek the solution of the coupled transport-VEF moment equations. For brevity, we use the drift-diffusion form of the VEF equations. The coupled system is: 
	\begin{subequations}
	\begin{equation}
		\Omegahat\cdot\nabla\psi + \sigma_t \psi = \frac{\sigma_s}{4\pi}\varphi + q \,,  
	\end{equation}
	\begin{equation}
		-\nabla\cdot\frac{1}{\sigma_t}\nabla\cdot\paren{\E\varphi} + \sigma_a \varphi = Q_0 - \nabla\cdot\frac{\Qone}{\sigma_t} \,.
	\end{equation}
	\end{subequations}
Boundary conditions are specified in Eqs.~\ref{vef:inflow} and \ref{vef:mlbc} for the transport and VEF drift-diffusion equations, respectively. Here, the transport equation's scattering source is coupled to the VEF drift-diffusion equation and the VEF drift-diffusion equation is nonlinearly coupled to the transport equation through the Eddington tensor and boundary factor. We have increased the complexity of the problem by adding the VEF scalar flux as an additional unknown and by casting the linear transport problem as nonlinear. However, properties of the VEF data and the transport equation allow this coupled, nonlinear system to be solved for efficiently than the transport equation in isolation. 

Let, 
	\begin{equation}
		\mat{L}\psi = \Omegahat\cdot\nabla\psi + \sigma_t \psi \,,
	\end{equation}
	\begin{equation}
		\mat{R}(\psi)\varphi = -\nabla\cdot\frac{1}{\sigma_t}\nabla\cdot\paren{\E(\psi)\varphi} + \sigma_a \varphi 
	\end{equation}
be the streaming and collision and VEF drift-diffusion operators, respectively, where $\paren{\cdot}$ indicates a nonlinear dependence on the argument. Furthermore, let $Q = Q_0 - \nabla\cdot\frac{\Qone}{\sigma_t}$ be the source for the VEF drift-diffusion equation. The coupled transport-VEF system can then be written 
	\begin{subequations}
	\begin{equation}
		\mat{L} \psi = \frac{\sigma_s}{4\pi} \varphi + q \,, 
	\end{equation}
	\begin{equation}
		\mat{R}(\psi)\varphi = Q \,,
	\end{equation}
	\end{subequations}
By linearly eliminating the angular flux, the coupled system is equivalent to 
	\begin{equation} \label{vef:lin_elim}
		\mat{R}\!\paren{\mat{L}^{-1}\!\paren{\frac{\sigma_s}{4\pi}\varphi + q}}\varphi = Q \,. 
	\end{equation}
Applying the inverse of the drift-diffusion operator, the solution of the coupled transport-VEF system is the fixed-point 
	\begin{equation} \label{vef:fixedpoint}
		\varphi = G(\varphi) \,,
	\end{equation}
where 
	\begin{equation}
		G(\varphi) = \mat{R}\!\paren{\mat{L}^{-1}\!\paren{\frac{\sigma_s}{4\pi}\varphi + q}}^{-1}Q \,. 
	\end{equation}
The fixed-point operator $G$ is applied in two stages: 1) solve the transport equation using a scattering source formed from the VEF scalar flux and 2) solve the VEF drift-diffusion equation using the VEF data computed from the angular flux from stage 1). 
Note that eliminating the angular means it appears only as an auxiliary variable used to computed the VEF data
This iteration is depicted in Fig.~\ref{vef:vef_alg}. 
% --- depiction of coupling between VEF and transport --- 
\begin{figure}
\centering
\includegraphics{figs/vef_alg.pdf}
\caption{A depiction of the iteration scheme used in VEF algorithms. The transport equation informs the VEF system through the VEF data while the VEF equations drive the transport equation through computation of the scattering source. By lagging the scattering term, the transport equation can be efficiently inverted. Rapid convergence occurs because the VEF data are weak functions of the solution.}
\label{vef:vef_alg}
\end{figure}

The simplest algorithm to solve Eq.~\ref{vef:fixedpoint} is fixed-point iteration: 
	\begin{equation}
		\varphi^{k+1} = G(\varphi^k) 
	\end{equation}
where $\varphi^0$ is an initial guess. This process is repeated until the difference between successive iterates is small enough. 

Iterative efficiency can be improved with the use of Anderson acceleration. Anderson acceleration defines the next iterate as the linear combination of the previous $m$ iterates that minimizes the residual $\varphi - G(\varphi)$. For the storage cost of $m$ previous iterates, Anderson acceleration increases the convergence rate and improves robustness. While it is not practical to store multiple copies of the \emph{angular} flux, it is reasonable to expect that a small set of \emph{scalar} flux-sized vectors can be stored. The process of linearly eliminating the transport equation, codified in Eq.~\ref{vef:lin_elim}, allows the Anderson space to be built from the much smaller scalar flux-sized vectors only. In the case where a subset of the angular flux unknowns are not eliminated, such as when a parallel block Jacobi sweep is used to avoid communication costs or when mesh cycles or reentrant faces are present, the solution vector can be augmented with these un-eliminated unknowns so that they are included in the Anderson space. This is the nonlinear analog to the ideas used for Krylov-accelerated source iteration \cite{doi:10.13182/NSE02-14}. 

In addition, defining the nonlinear residual as 
	\begin{equation}
		F(\varphi) = \varphi - G(\varphi) = 0 \,,
	\end{equation}
root-finding methods such as \gls{jfnk} can be used. JFNK builds a new Krylov space to approximate the gradient of $F$ \emph{at each iteration} meaning information across iterations is not kept. JFNK typically required significantly more evaluations of $G$ than Anderson-accelerated fixed-point iteration. Since evaluating $G$ involves inverting the transport equation, this significantly increases the expense of the algorithm. Thus, we present results using fixed-point iteration and Anderson-accelerated fixed-point iteration only. 

\section{Bounds and Asymptotic Limits of the VEF Data}
The Eddington tensor and boundary factor are defined as 
	\begin{equation}
		\E = \frac{\int\Omegahat\otimes\Omegahat\,\psi \ud \Omega}{\int \psi \ud \Omega}\,,
	\end{equation}
	\begin{equation}
		E_b = \frac{\int|\Omegahat\cdot\n|\,\psi \ud \Omega}{\int \psi \ud \Omega} \,,
	\end{equation}
respectively. Observe that the Eddington tensor and boundary factor are $\psi$-weighted averages of $\Omegahat\otimes\Omegahat$ and $|\Omegahat\cdot\n|$, respectively. This means the VEF data are bounded functions of $\psi$. 
Since the Eddington tensor is a $\psi$-weighted average of $\Omegahat\otimes\Omegahat$, its maximum and minimum occur at the maximum and minimum of $\Omegahat\otimes\Omegahat$. This can be seen by setting $\psi$ to be a Dirac delta function centered at an extreme value of $\Omegahat\otimes\Omegahat$. Due to to this, the Eddington tensor obeys the bounds 
	\begin{equation}
		\E_{ij} \in \begin{cases}
			[0,1] \,, & i=j \\ 
			[-1/2,1/2] \,, & i\neq j
		\end{cases} \,. 
	\end{equation}
Likewise, the boundary factor has the extreme values of $|\Omegahat\cdot\n|$. Thus, 
	\begin{equation}
		E_b \in [0,1] \,. 
	\end{equation}

In the thick diffusion limit, the angular flux is a linearly anisotropic function in angle. In other words, for some spatially-dependent functions $f(\x)$ and $\vec{g}(\x)$, the angular flux is of the form: 
	\begin{equation}
		\psi(\x,\Omegahat) = \frac{1}{4\pi}\paren{f(\x) + \Omegahat\cdot\vec{g}(\x)} \,. 
	\end{equation}
The zeroth and second moments of this linearly anisotropic solution are 
	\begin{equation}
		\int \psi \ud \Omega = \frac{1}{4\pi} \int f(\x) + \Omegahat\cdot\vec{g}(\x) \ud \Omega = f(\x) \,, 
	\end{equation}
	\begin{equation}
		\int \Omegahat\otimes\Omegahat\,\psi \ud \Omega = \frac{1}{4\pi}\int \Omegahat\otimes\Omegahat\paren{f(\x) + \Omegahat\cdot\vec{g}(\x)} \ud \Omega = \frac{f(\x)}{3}\I \,, 
	\end{equation}
since integrals of odd functions in angle over the unit sphere are zero. Thus, in the thick diffusion limit, the Eddington tensor is 
	\begin{equation}
		\E = \frac{f(\x)/3\I}{f(\x)} = \frac{1}{3}\I \,. 
	\end{equation}
For the boundary factor, 
	\begin{equation}
		\int |\Omegahat\cdot\n|\,\psi \ud \Omega = \frac{1}{4\pi}\int |\Omegahat\cdot\n|\paren{f(\x) + \Omegahat\cdot\vec{g}(\x)} \ud \Omega = \frac{f(\x)}{2} \,, 
	\end{equation}
and thus 
	\begin{equation}
		E_b = \frac{f(\x)/2}{f(\x)} = \frac{1}{2} 
	\end{equation}
in the thick diffusion limit. 
Therefore, the VEF drift-diffusion equation with Miften-Larsen boundary conditions limits to the radiation diffusion equation with Marshak boundary conditions: 
	\begin{subequations}
	\begin{equation}
		-\nabla\cdot\frac{1}{3\sigma_t}\nabla\varphi + \sigma_a \varphi = Q \,, \quad \x \in \D \,, 
	\end{equation}
	\begin{equation}
		\vec{J}\cdot\n = \frac{1}{2}\varphi + 2\Jin \,, \quad \x \in \partial \D\,. 
	\end{equation}
	\end{subequations}
	
\section{Functional Derivatives of the VEF Data} \label{vef_sec:gateaux}
\subsection{The Gateaux Derivative}
The Gateaux derivative is a generalization of the directional derivative that supports derivatives of functionals (i.e.~a function whose argument is a function) as well as complicated mathematical objects such as second-order tensors \cite{vainberg64}. Let $f : U\rightarrow V$ be a (possibly nonlinear) mapping from a space $U$ to a space $V$. For example, a simple scalar function $f(\x)$ would set $U = \R^{\dim}$ and $V = \R$. The Gateaux derivative of $f$ evaluated at $u\in U$ in the direction $v \in V$ is given by 
	\begin{equation} \label{vef:gateaux_limit}
		D[f](u,v) = \lim_{\omega\rightarrow 0} \frac{f(u+\omega v) - f(u)}{\omega} \,,
	\end{equation}
where $\omega\in \R$. Since 
	\begin{equation}
	\begin{aligned}
		\bracket{\pderiv{}{\omega}f(u + \omega v)}_{\omega=0} &= \bracket{\lim_{\Delta\omega \rightarrow 0} \frac{f(u+(\omega+\Delta\omega)v) - f(u + \omega v)}{\Delta\omega}}_{\omega=0} \\
		&= \lim_{\Delta\omega \rightarrow 0}\bracket{\frac{f(u + \omega v + \Delta\omega v) - f(u + \omega v)}{\Delta\omega}}_{\omega=0} \\
		&= \lim_{\Delta\omega\rightarrow 0} \frac{f(u + \Delta\omega v) - f(u)}{\Delta\omega} \,,
	\end{aligned}
	\end{equation}
where continuity of $f$ is used to move the limit outside of the brackets, the Gateaux derivative can also be equivalently be written 
	\begin{equation} \label{vef:gateaux}
		D[f](u,v) = \bracket{\pderiv{}{\omega}f(u + \omega v)}_{\omega=0} \,. 
	\end{equation}
We favor Eq.~\ref{vef:gateaux} over Eq.~\ref{vef:gateaux_limit} as it leads to simpler calculations by leveraging the existing machinery of the partial derivative. As an example, if $u : \R^2 \rightarrow \R$ and $\vec{v} : \R^2 \rightarrow \R^2$, we can compute $\vec{v}\cdot\nabla u|_{\x}$ using the above definition as 
	\begin{equation}
		D[u](\x,\vec{v}) = \pderiv{}{\omega}\bracket{u(\x + \omega\vec{v})}_{\omega = 0} \,. 
	\end{equation}
To particularize, let $u(\x) = xy$ and $\vec{v} = \vector{v_1 & v_2}$, then 
	\begin{equation}
	\begin{aligned}
		D[u](\x,\vec{v}) &= \pderiv{}{\omega}\bracket{(x+\omega v_1)(y + \omega v_2)}_{\omega = 0} \\
		&= \pderiv{}{\omega}\bracket{xy + \omega(xv_2 + yv_1) + \omega^2 v_1 v_2}_{\omega = 0} \\
		&= \bracket{xv_2 + yv_1 + 2\omega v_1 v_2}_{\omega = 0} \\
		&= x v_2 + y v_1 \\ 
		&= \vec{v} \cdot \nabla (xy) \,. 
	\end{aligned}
	\end{equation}
This establishes the connection between the directional derivative and the Gateaux derivative. 

In addition, in the context of a Newton method, the Gateaux derivative defines a systematic method for computing the action of the Jacobian. Consider the first-order Taylor series expansion of a function $f$ about $\x_0$: 
	\begin{equation}
		f(\x) \xrightarrow{\text{TSE}} f(\x_0) + \pderiv{f}{\x}\biggr\rvert_{\x_0}(\Delta \x) \,,
	\end{equation}
where $\Delta \x = \x - \x_0$. That is, the function $f$ is evaluated at $\x_0$ and the gradient of $f$ evaluated at $\x_0$ in the direction of $\Delta\x$ is added to form a linear approximation to $f$. Thus, we can alternatively write 
	\begin{equation}
		f(\x) \xrightarrow{\text{TSE}} f(\x_0) + D[f](\x_0, \Delta \x) \,. 
	\end{equation}
In this way, the Gateaux derivative provides a process for linearizing any $f$ even when $f$ is a tensor-valued functional and the argument $\x$ is itself a function. For example, we can linearize the Eddington tensor about some angular flux $\psi_0$ using: 
	\begin{equation}
		\E(\psi) \xrightarrow{\text{TSE}} \E(\psi_0) + D[\E](\psi_0,\psi') \,. 
	\end{equation}
This linearization is used in this section to investigate the properties of the Eddington tensor and is also used in Chapter \ref{chap:smm} to derive discrete \glspl{smm}. 

\subsection{Derivation of Functional Derivatives}
Applying the definition in Eq.~\ref{vef:gateaux} to the Eddington tensor, the derivative of the Eddington tensor evaluated at $\psi_0$ in the direction $\psi'$ is 
	\begin{equation}
	\begin{aligned}
		D[\E](\psi_0,\psi') &= \pderiv{}{\omega} \bracket{\E(\psi_0 + \omega\psi')}_{\omega=0} \\
		&= \pderiv{}{\omega}\bracket{\frac{\int \Omegahat\otimes\Omegahat\paren{\psi_0 + \omega\psi'}}{\int \psi_0 + \omega \psi' \ud \Omega}}_{\omega=0} \\
		&= \pderiv{}{\omega}\bracket{\frac{\P_0 + \omega\P'}{\phi_0 + \omega\phi'}}_{\omega=0} \,,
	\end{aligned}
	\end{equation}
where $\phi_0$ and $\P_0$ are the zeroth and second moments of $\psi_0$ and $\phi'$ and $\P'$ the zeroth and second moments of $\psi'$. Applying the quotient rule, 
	\begin{equation}
	\begin{aligned}
		\pderiv{}{\omega}\bracket{\frac{\P_0 + \omega\P'}{\phi_0 + \omega\phi'}}_{\omega=0} &= \frac{\P'(\phi_0 + \omega \phi') - (\P_0 + \omega\P')\phi'}{(\phi_0 + \omega\phi')^2} \biggr\rvert_{\omega=0} \\
		&= \frac{\P' \phi_0 - \P_0 \phi}{\phi_0^2} \\
		&= \frac{1}{\phi_0}\paren{\P' - \frac{\P_0}{\phi_0} \phi'} \\
		&= \frac{1}{\phi_0}\paren{\P' - \E_0 \phi'} \,, 
	\end{aligned}
	\end{equation}
where $\E_0 = \P_0/\phi_0$ is the Eddington tensor evaluated at $\psi = \psi_0$. 
Thus, the derivative of the Eddington tensor evaluated at $\psi_0$ in the direction $\psi'$ is: 
	\begin{equation} \label{vef:Edd_deriv}
		D[\E](\psi_0,\psi') = \frac{1}{\phi_0} \paren{\int \Omegahat\otimes\Omegahat\,\psi'\ud\Omega - \E_0\int \psi' \ud \Omega} \,. 
	\end{equation}
Note that $D[\E](\psi_0,\psi')$ is also a second-order tensor. The above process applies analogously to the boundary factor. 
The Gateaux derivative of the boundary factor at $\psi_0$ in the direction $\psi'$ is 
	\begin{equation} \label{vef:Eb_deriv}
	\begin{aligned}
		D[E_b](\psi_0,\psi') &= \pderiv{}{\omega}\bracket{E_b(\psi + \omega\psi')}_{\omega=0} \\ 
		&= \pderiv{}{\omega}\bracket{\frac{\int |\Omegahat\cdot\n|\paren{\psi_0 + \omega\psi'} \ud \Omega}{\int \psi_0 + \omega\psi' \ud \Omega}}_{\omega=0} \\
		&= \frac{\paren{\int \psi_0 + \omega\psi' \ud \Omega}\paren{\int |\Omegahat\cdot\n|\,\psi'\ud\Omega} - \paren{\int |\Omegahat\cdot\n|\paren{\psi_0 + \omega\psi'}\ud\Omega}\paren{\int\psi'\ud\Omega}}{\paren{\int \psi_0 + \omega\psi' \ud \Omega}^2} \biggr\rvert_{\omega=0} \\
		&= \frac{1}{\phi_0}\bracket{\int |\Omegahat\cdot\n|\,\psi' \ud \Omega - E_{b0} \int \psi' \ud \Omega} \,,
	\end{aligned}
	\end{equation}
where $E_{b0} = \int |\Omegahat\cdot\n|\,\psi_0 \ud \Omega / \int \psi_0 \ud \Omega$. 

\subsection{Intuition for the Rapid Convergence of the VEF Algorithm}
The rapid convergence of VEF algorithms is due to the VEF data having weak dependence on the angular flux as characterized by having small functional derivatives with respect to the solution \cite{goldin}. Too see this, consider the linearization of the Eddington tensor about the previous iteration's angular flux, $\psi^\ell$: 
	\begin{equation} \label{vef:edd_lin}
		\E(\psi) \approx \E(\psi^\ell) + D[\E](\psi^\ell,\psi') \,. 
	\end{equation}
If we set $\psi'$ to be the error at iteration $\ell$ such that $\psi' = \psi - \psi^\ell$ with $\psi$ the solution of the transport problem, the above linearization provides an approximation for how the Eddington tensor will change as the algorithm proceeds to the next iteration. Note that the VEF algorithm converges when the VEF data converge. Thus, if the size of $D[E](\psi^\ell,\psi - \psi^\ell)$ is small enough relative to the iteration's stopping tolerance, the Eddington tensor will change by an amount small enough to allow the iteration to terminate. 

We now show three examples of pairs of evaluation points, $\psi_0$, and directions, $\psi'$, where the Gateaux derivative is zero. In these cases, the Eddington tensor evaluated at $\psi_0$ is an approximation to the true Eddington tensor to the accuracy of the linearization process (e.g.~$\mathcal{O}(\psi')^2$). 
First, consider the direction being a scalar multiple of $\psi_0$ such that $\psi' = \alpha \psi_0$ for some $\alpha \in \R$. The Gateaux derivative for this case is: 
	\begin{equation}
		D[\E](\psi_0,\alpha\psi_0) = \frac{1}{\phi_0}\paren{\int \Omegahat\otimes\Omegahat\,\alpha\psi_0 \ud\Omega - \E_0 \int \alpha \psi_0 \ud \Omega} = \frac{\alpha}{\phi_0}\paren{\P_0 - \P_0} = 0 \,,
	\end{equation}
since $\E_0 \int \psi_0 \ud \Omega = \P_0$. 
Additionally, if we add a perturbation to $\psi_0$ that is linear in angle, 
	\begin{equation}
	\begin{aligned}
		D[\E](\psi_0, \psi_0 + \Omegahat\cdot\vec{g}(\x)) &= \frac{1}{\phi_0}\paren{\int \Omegahat\otimes\Omegahat\paren{\psi_0 + \Omegahat\cdot\vec{g}(\x)}\ud \Omega - \E_0 \int \psi_0 + \Omegahat\cdot\vec{g}(\x) \ud \Omega}\\
		&= \frac{1}{\phi_0} \paren{\P_0 - \P_0} \\
		&= 0 \,. 
	\end{aligned}
	\end{equation}
The above holds for any spatially-dependent function, $\vec{g}(\x)$, and also for any perturbation that is odd in angle. 
Finally, let $\psi_0$ and $\psi'$ be linearly anisotropic in angle such that $\psi_0 = f_0(\x) + \Omegahat\cdot\vec{g}_0(\x)$ and $\psi' = f'(\x) + \Omegahat\cdot\vec{g}'(\x)$, then 
	\begin{equation}
	\begin{aligned}
		D[\E](f_0(\x) + \Omegahat\cdot\vec{g}_0(\x), f'(\x) + \Omegahat\cdot\vec{g}'(\x)) &= \frac{1}{4\pi f_0}\paren{\frac{4\pi f'}{3}\I - \frac{1}{3}\I \cdot 4\pi f'} \\
		&= \frac{f'}{f}\paren{\frac{1}{3}\I - \frac{1}{3}\I} \\
		&= 0 \,. 
	\end{aligned}
	\end{equation}
Note that the $D[E_b](\psi_0,\psi') = 0$ for each of the pairs $(\psi_0,\psi')$ discussed above as well. 
Thus, if the error at any iteration is a scalar multiple or an odd-in-angle perturbation of the current iteration's solution, the functional derivatives of the VEF data are zero. This is also true if the current iteration's solution is linearly anisotropic and the true solution is linearly anisotropic. 

\section{Mathematical Properties of the VEF Equations}
The presence of the Eddington tensor inside the divergence in the VEF first moment equation leads to cross derivative terms not present in the standard form of the radiation diffusion equation. This can be seen in the differential term in the drift-diffusion form of the VEF equations given by
	\begin{equation}
		\nabla\cdot\frac{1}{\sigma_t}\nabla\cdot\paren{\E\varphi} \,. 
	\end{equation}
Assuming the Eddington tensor and total cross section have the required differentiability, we can use the product rule to write 
	\begin{equation}
	\begin{aligned}
		\nabla\cdot\frac{1}{\sigma_t}\nabla\cdot\paren{\E\varphi} &= \nabla\cdot\frac{\E}{\sigma_t}\nabla\varphi + \nabla\cdot\paren{\frac{\nabla\cdot\E}{\sigma_t}\varphi} \\
		&= \nabla\cdot\frac{\E}{\sigma_t}\nabla\varphi + \frac{\nabla\cdot\E}{\sigma_t}\cdot\nabla\varphi + \paren{\nabla\cdot\frac{\nabla\cdot\E}{\sigma_t}} \varphi \,. 
	\end{aligned}
	\end{equation}
Defining 
	\begin{equation}
		\mat{D} = \frac{\E}{\sigma_t}\,,\quad \vec{c} = -\frac{\nabla\cdot\E}{\sigma_t}\,, \quad \gamma = \nabla\cdot\frac{\nabla\cdot\E}{\sigma_t} \,,
	\end{equation}
the VEF drift-diffusion equation can be written as the diffusion-advection-reaction equation: 
	\begin{equation} \label{vef:standard_elliptic}
		-\nabla\cdot\mat{D}\nabla\varphi + \vec{c}\cdot\nabla\varphi + (\sigma_a - \gamma) \varphi = Q \,. 
	\end{equation}
Here, it is clear that the VEF drift-diffusion equation is not symmetric due to the presence of the advective term, $\vec{c}\cdot\nabla\varphi$. In addition, since $\mat{D}$ is symmetric positive definite (since the Eddington tensor is symmetric positive definite), the VEF drift-diffusion equation is an elliptic partial differential equation. 

However, since the transport equation allows discontinuous solutions in space and angle, the Eddington tensor is generally not differtiable in space. Numerically, it is common to use a \gls{dg} spatial discretization for the transport equation. In such case, the solution is generally discontinuous across interior mesh interfaces. Thus, the VEF drift-diffusion equation cannot be written in the standard elliptic form of Eq.~\ref{vef:standard_elliptic} since the Eddington tensor does not have the required regularity to have $\nabla\cdot\E$ be well defined. This means discretization techniques must be extended to handle the non-standard form of the VEF drift-diffusion equation. 

\section{Discrete VEF Schemes}
The goal of this dissertation is to evaluate 
	\begin{equation}
		G(\varphi) = \mat{R}\!\paren{\mat{L}^{-1}\!\paren{\frac{\sigma_s}{4\pi}\varphi + q}}^{-1} Q \,. 
	\end{equation}
numerically in a computationally efficient manner. Numerically approximating this operator requires defining 1) a discretization of the transport equation, 2) a representation for the Eddington tensor and boundary factor, and 3) a discretization for the VEF moment system. Discrete VEF methods are generally classified as consistent and independent based on their approach for defining the algorithmic choices corresponding to 2) and 3). Consistent methods are characterized by having discrete transport and VEF equations that are algebraically consistent. These methods produce solutions for the moment system that match the moments of the discrete transport equation to machine precision (or the minimum of the solver tolerances). In other words, the difference between the moments of the transport solution and the solution of the VEF moment system differ in a manner that is independent of the mesh size. Consistent methods are derived by forming moment equations from the \emph{discrete} transport equation and applying the VEF closures. This leads to a VEF discretization that is an equivalent reformulation of the discrete transport equation leading to the discrete equivalence that characterizes these methods. 

On the other hand, independent VEF methods are characterized as having discrete transport and VEF equations that are not algebraically consistent. Due to this, the moments of the discrete transport solution and the VEF solution differ on the order of the spatial discretization error and are thus only equivalent in the limit as the spatial mesh is refined. Independent methods are derived by discretizing the continuous VEF equations without regard for the discretization used for the transport equation. Rapid convergence is maintained when the discrete VEF data are represented in a sufficiently consistent manner. 

Figure \ref{vef:commuting} depicts a commuting diagram for approximating the continuous transport problem with a discrete VEF moment system. The consistent approach moves down and to the right from the continuous transport equation corresponding to discretizing and using discrete moments and closures to form the VEF system. Independent methods move right and then down corresponding to forming the continuous VEF equations and then discretizing. It is important to note that both approaches lead to discrete VEF systems that approximate the continuous transport problem. However, only the consistent approach produces a method with an equivalence in the bottom row of the commuting diagram.
% --- VEF commuting diagram --- 
\begin{figure}
\centering
\includegraphics[width=.85\textwidth]{figs/vef_commuting.pdf}
\caption{The commuting diagram for VEF approximations of the transport equation. Consistent VEF methods apply discrete closures to a discretized transport equation whereas the independent methods discretize the continuous VEF equations formed through continuous closures of the continuous transport equation. In both cases, the discrete VEF system is an approximation to the continuous transport equation. However, independent methods generally do not have equivalence of the bottom row. That is discrete VEF and transport are in general not equivalent.}
\label{vef:commuting}
\end{figure}

Consistent methods are attractive in that they can be used in place of an existing transport method without changing the solution. This is particularly important in reactor physics applications where licensing restrictions may require that new methods exactly reproduce solutions produced by older methods. In addition, forming the discrete VEF system from the discrete transport equation provides a systematic process for steps 2) and 3) of the VEF method. While this provides a straightforward process for defining a consistent VEF method given a discrete transport equation, this often leads to VEF discretizations that are difficult to solve efficiently with existing linear solver technology. Furthermore, use of negative flux fixups to ensure positivity of the transport equation will render an otherwise consistent VEF method inconsistent. 

By contrast, the independent approach allows significant algorithmic flexibility. 
\textcite{two-level-independent-warsa} compared the iterative efficiency of consistent and independent methods and saw equivalent convergence as long as the independent method properly represented the VEF data. In particular, using \gls{sn} angular quadrature and finite element interpolation produced independent methods that converged as rapidly as a consistent method. \todo{more benefits, might need to wait until intro is written}

% --- benefits --- 
% automatic TDL and conservation even if transport doesn't 
% direct coupling to multiphysics 
% solvers 
% optimal design for each separately 

\end{document}