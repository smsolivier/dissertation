%!TEX root = ../doc.tex
\documentclass[../doc.tex]{subfiles}

\begin{document}
I am deeply grateful for the support and guidance of many people. 
Foremost, I would like to thank Dr.~Terry Haut, my external advisor from Lawrence Livermore National Laboratory, for his expertise, mentorship, endless encouragement, and infectious enthusiasm for all things math. I would not be the confident mathematician I am today without his calming guidance or our many discussions over the past four years. By Terry's example, I learned to value and enjoy collaboration almost as much as the technical details. 

I would also like to thank the members of the ``VEF Biweekly Meeting'', Professor Dmitriy Anistratov (North Carolina State University) and Drs.~Terry Haut, Ben Yee (LLNL), Ben Southworth (LANL), and James Warsa (LANL), who helped sustain three years of productive and insightful meetings that often acted as the sounding board for the ideas presented here. Their expertise saved me countless hours. In particular, I want to thank Professor Anistratov for lending his immense experience with the topics discussed in this dissertation and for his encyclopedic ability to always know a reference to a related idea in the literature. 

I am grateful to the ``LDRD code team'' at LLNL consisting of Drs.~Terry Haut, Ben Yee, and Milan Holec, for welcoming me into their productive research environment. Their related efforts, with Ben Yee in particular, expanded the reach, impact, and scope of my work. I would also like to thank LLNL for their generous computing allowance and the MFEM team for providing the software foundation that inspired me to learn as much as I could about finite elements and made the implementation of the enclosed work enjoyable.  

My sincere gratitude goes to Professor Jim Morel at Texas A\&M University for introducing me to this topic as part of an undergraduate research course. 
I am truly grateful for his guidance in the earliest stages of my professional development and for identifying the fruitful and engaging research problem that ultimately culminated in this dissertation and the many skills and connections I gained along the way. 
Jim also brought the second moment method to my attention in my final year of graduate school, which I consider to be the most exciting and promising outcome of my work. 

I am very fortunate to have been supported by a Department of Energy Computational Science Graduate Fellowship which afforded me the freedom and autonomy to pursue this research and collaborate so broadly. 
I am also thankful for the Applied Science \& Technology program at UC Berkeley which provided the flexibility to explore all my passions and interests regardless of department or college and design a unique degree I am proud to have. I am truly lucky to have been able to have this experience and sincerely appreciate all those named above (and many more!) for their part in my success. 

% Finally, thanks to Professor Per-Olof Persson for taking me on as a student in my final semester and serving on my qualifying and dissertation committees. 
\end{document}