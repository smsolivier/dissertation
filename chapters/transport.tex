%!TEX root = ../doc.tex
\documentclass[../doc.tex]{subfiles}

\begin{document}
\chapter{Radiation Transport Background}
\section{The Angular Flux and Its Moments}
\section{The Equations of Thermal Radiative Transfer}
\section{Steady-State, Linear Transport: A Proxy for TRT}
\section{Useful Angular Identities}
\section{The Radiation Diffusion Approximation}

% --- phase diagram --- 
\begin{figure}
\centering
\includegraphics[width=.85\textwidth]{figs/phase.pdf}
\caption{A depiction of radiation transport's seven-dimensional phase space. }
\label{trans:phase}
\end{figure}

% --- (x,Omega) diagram --- 
\begin{figure}
\centering
\includegraphics[width=.85\textwidth]{figs/omega_space.pdf}
\caption{A depiction of the direction-of-flight portion of the phase space at a fixed location $\x$, frequency $\nu$, and time $t$. }
\label{trans:omega_space_diag}
\end{figure}

% --- spherical coordinates for Omega --- 
\begin{figure}
\centering
\includegraphics[width=.65\textwidth]{figs/omega.pdf}
\caption{A depiction of the spherical coordinate system used for the direction of particle travel variable, $\Omegahat$. Here, $\theta \in [0,2\pi]$ is the azimuthal angle and $\varphi \in [0,\pi]$ the polar angle. }
\label{trans:omega_diagram}
\end{figure}

\end{document}