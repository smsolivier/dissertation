%!TEX root = ../doc.tex
\documentclass[../doc.tex]{subfiles}

\begin{document}
\chapter{Radiation Transport Background} \label{chap:transport}
The purpose of this chapter is to motivate and define the simplified transport problem that serves as the model for the development of the moment methods derived in later chapters. To that end, the general kinetic Boltzmann equation is introduced and then particularized to the equations of thermal radiative transfer which describe thermal photons in the absence of material motion and heat conduction. We pay particular attention to the notational differences present in the fields of nuclear engineering, astrophysics, and \gls{hedp}. The simplifications of isotropic scattering and frequency and time-independence are applied to derive the model problem. The chapter concludes with a derivation of the radiation diffusion approximation. 

\section{The Boltzmann Transport Equation}
A discussion of the Boltzmann transport equation must begin with the definition of the particle distribution function, $f$. 
The set of all possible positions $\x$ and velocities $\v$ is called the system's phase space. The particle distribution function represents the expected number of particles at each point in the phase space at a given time. In other words, $f(\x,\v,t) \ud \x \ud \v = $ the expected number of particles in the differential phase space volume at a time $t$. 

The Boltzmann transport equation describes the evolution of the particle distribution function over time. In general, the Boltzmann equation is written as:
	\begin{equation}
		\dderiv{f}{t} = \paren{\pderiv{f}{t}}_\text{force} + \paren{\pderiv{f}{t}}_\text{collision} \,,
	\end{equation}
where the force and collision terms are application-dependent sources or sinks for the particle distribution function \cite[Chapter 7]{chen2012introduction}. The force term represents the effect of an external influence (i.e.~not caused by the particles themselves) and the collision term represents the effect of particles colliding with each other or a background material. If a force $\F$ acts on the particles over a time period of $\ud t$, the particles' position and velocity will change by an amount $\ud \x = \v\ud t$ and $\ud \v = \frac{\F}{m} \ud t$ where $m$ is the mass of the particle. Taking the total derivative of $f$ in the phase space: 
	\begin{equation}
	\begin{aligned}
		\ud f &= \pderiv{f}{t} \ud t + \nabla f \cdot\ud \x + \nabla_{\v} f \cdot\ud \v \\
		&= \pderiv{f}{t} \ud t + \nabla f \cdot\v \ud t + \nabla_{\v}f \cdot\frac{\F}{m} \ud t \,, 
	\end{aligned}
	\end{equation}
where $\nabla_{\v} = \pderiv{}{v_x} \e_x + \pderiv{}{v_y}\e_y + \pderiv{}{v_z} \e_z$ denotes the gradient in velocity space. 
Dividing by $\ud t$ yields: 
	\begin{equation}
		\dderiv{f}{t} = \pderiv{f}{t} + \v \cdot \nabla f + \frac{\F}{m} \cdot \nabla_{\v} f \,. 
	\end{equation}
Thus, the force term $\paren{\pderiv{f}{t}}_\text{force} = \frac{\F}{m}\cdot\nabla_{\v} f$ so that the general form of the Boltzmann equation can be equivalently be written: 
	\begin{equation}
		\pderiv{f}{t} + \v \cdot \nabla f + \frac{\F}{m} \cdot \nabla_{\v} f = \paren{\pderiv{f}{t}}_\text{collision} \,. 
	\end{equation}
For charged particle transport, the particles experience external electromagnetic forces and frequently interact with each other through Coulomb collisions. In this case, particles with charge $q$ experience the force $\F = q(\E + \v\times \mat{B})$ due to the electric field, $\E$, and magnetic field, $\mat{B}$. The Coulombic collision term is described by the (quite complicated) Fokker-Planck operator \cite[Eq.~7.25]{chen2012introduction}. 

Here, we focus on neutral particle transport (i.e.~photons or neutrons). Since these particles do not have charge, there are no external electromagnetic forces. Furthermore, particle-particle collisions are exceedingly rare and it is thus commonplace to ignore them \cite{neutron_transport_LM}. The collision term then only includes the interaction of the neutral particle with the background medium. These interactions are broadly classified into absorption and scattering events. The probabilities of these events occurring per unit distance traveled are called cross sections for neutrons and opacities for photons and are governed by nuclear physics. The most general form of the Boltzmann transport equation for a neutral particle is then: 
	\begin{equation} \label{trans:linboltz}
		\pderiv{f}{t} + \v \cdot \nabla f = \paren{\pderiv{f}{t}}_\text{absorption} + \paren{\pderiv{f}{t}}_\text{scattering} + \paren{\pderiv{f}{t}}_\text{source} \,,
	\end{equation}
where we have split the collision term into absorption and scattering terms and included an additional term representing an external source of particles. For neutrons, the external source could be a fixed-source of particles such as a radioactive material that emits neutrons. For photons, we consider the source to be the thermal emission of photons. Thus, the photon external source is dependent on the material's temperature. We note that the case of the general Boltzmann equation under the assumption that particles do not interact with each other is called the linear Boltzmann equation. 

\section{Boundary and Initial Conditions}
Suppose that the Boltzmann transport equation is solved in a spatial domain $\D$ with boundary $\partial\D$. To solve the transport equation, the initial and boundary conditions for the particle distribution function must be provided. The initial condition corresponds to specifying $f(\x,\v,0)$: the expected number of particles in the phase space at time $t=0$. Letting $\n$ represent the outward unit normal to the boundary of the domain, the boundary condition specifies $f(\x,\v,t)$ for each $\x\in\partial\D$ satisfying $\v\cdot\n<0$ for all values of $t$. Figure \ref{trans:inflow_diag} depicts an example domain and its boundary along with a selection of normal vectors. The region $\v\cdot\n<0$ where the boundary conditions must be provided are shown in red. The case where 
	\begin{equation} \label{trans:gen_inflow}
		f(\x,\v,t) = \bar{f}(\x,\v,t) \,, \quad \x \in \partial\D \ \mathrm{and} \ \v \cdot \n < 0 \,, 
	\end{equation}
is called an inflow boundary condition with $\bar{f}$ the inflow boundary function. The special case $\bar{f} \equiv 0$ is called a vacuum boundary condition. Other types of boundary conditions commonly used in radiation transport are provided in \cite[\S 1.3]{neutron_transport_LM}.
% --- inflow boundary conditions diagram --- 
\begin{figure}
\centering
\includegraphics[width=.85\textwidth]{figs/bean.pdf}
\caption{A kidney bean shaped domain depicted with a selection of normal vectors along its boundary. The inflow region corresponding to $\v\cdot\n<0$ is colored in red for two example directions of $\v$.}
\label{trans:inflow_diag}
\end{figure}

\section{Navigating Notation}
Application of the Boltzmann transport equation to modeling radiation is home to three sets of notation: nuclear engineering, astrophysics, and \gls{hedp}. Here, we define the notation used to describe the radiation field and its moments. The hope of this section is to provide the means for newcomers to this subject to navigate these three fields. 

\subsection{$f$, $\psi$, and $I$}
In radiation transport, the velocity variable is represented using a direction-of-flight variable, $\Omegahat$, and an energy variable. 
The angular variable, $\Omegahat$, is a direction on the unit sphere, $\mathbb{S}^2$, and is described using the spherical coordinate system depicted in Fig.~\ref{trans:spherical_coords} where polar and azimuthal angles are used to define the direction $\Omegahat$ in three-dimensional space. 
Figure \ref{trans:omega_phase} depicts the direction-of-flight portion of the phase space. Particles at any point in time, space, and frequency can travel in any direction on the unit sphere. The differential phase space volume associated with the angular variable is the cone $\ud\Omega$ about the direction $\Omegahat$. 
\begin{figure}
\centering
\begin{subfigure}{.49\textwidth}
	\centering
	\includegraphics[width=\textwidth]{figs/omega.pdf}
	\caption{}
	\label{trans:spherical_coords}
\end{subfigure}
\begin{subfigure}{.49\textwidth}
	\centering
	\includegraphics[width=\textwidth]{figs/omega_space.pdf}
	\caption{}
	\label{trans:omega_phase}
\end{subfigure}
\caption{(a) a depiction of the spherical coordinate system used for the direction of particle travel variable, $\Omegahat$. Here, $\theta \in [0,2\pi]$ is the azimuthal angle and $\varphi \in [0,\pi]$ the polar angle. (b) A depiction of the direction-of-flight portion of the phase space at a fixed location $\x$, frequency $\nu$, and time $t$. At each position $\x$, particles can travel in any direction on the unit sphere.}
\end{figure}

In the case of neutrons, the energy variable is $E = \frac{1}{2}mv^2$ where $v = |\v|$ is the speed of the particle. For photons, $E = h\nu$ with $h$ Planck's constant and $\nu$ the photon frequency. Since $h$ is a constant, the photon distribution is usually presented as a function of frequency. These distribution functions are related through: 
	\begin{equation}
		\iiint f(\x,\v,t) \ud \x \ud \v \ud t = \iiiint f(\x,\Omegahat,E,t) \ud \x \ud \Omega \ud E \ud t = \iiiint f(\x,\Omegahat,\nu,t) \ud \x \ud \Omega \ud \nu \ud t \,. 
	\end{equation}
That is, each representation of the velocity variable represents the same expected number of particles. 
Although $\Omegahat$ is a three-dimensional vector, it is defined in spherical coordinates using only two parameters. This is possible since $\Omegahat$ is a unit vector. 
Thus, $f(\x,\Omegahat,E,t)$ and $f(\x,\Omegahat,\nu,t)$ are both still functions of seven independent variables. 

For nuclear reactors, neutrons are the primary particle of interest and the goal is to understand the fission power produced by the system. To that end, the transport equation is cast in terms of the \emph{angular flux}, $\psi$. The angular flux represents the particle path length density in the phase space. In other words, $\psi \ud \x \ud \Omega \ud E \ud t$ represents the expected distance traveled by particles located in the phase space element $\ud \x \ud \Omega\ud E$ in the time interval $\ud t$. In relation to the distribution function, $f$, the angular flux is defined as the product of the particle speed and the distribution function: 
	\begin{equation}
		\psi(\x,\Omegahat,E,t) \equiv v f(\x,\Omegahat,E,t) \,. 
	\end{equation}
This definition is the natural choice for computing reaction rates. If $\sigma$ represents the probability of a neutron inducing a reaction per unit length traveled, $\sigma \psi \ud \x \ud \Omega\ud E \ud t$ represents the expected number of reactions induced by neutrons traveling in the phase space element $\ud\x\ud\Omega\ud E$ in the time $\ud t$. The reaction rate is computed by integrating $\sigma \psi$ over the domain of the reaction, $\D$, all angles on the unit sphere, $\mathbb{S}^2$, and all energies $E \in [0,\infty)$. Thus, the number of reactions per unit time is computed as 
	\begin{equation}
		\int_\D \int_{\mathbb{S}^2} \int_0^\infty \sigma \psi \ud E \ud \Omega \ud \x \,. 
	\end{equation}

Radiation transport in astrophysics and \gls{hedp} is often concerned with the energy transfer and deposition associated with thermal photons. The Boltzmann equation is then typically cast in terms of the \emph{intensity}, $I$. The intensity is defined with power units, energy per unit time, but is also expressed per unit area, per unit sold angle, and per unit frequency bandwidth. It is defined as 
	\begin{equation}
		I(\x,\Omegahat,\nu,t) \equiv c h \nu f(\x,\Omegahat,\nu,t) \,,
	\end{equation}
where $c$ is the speed of light. 
Since photons travel at the speed of light, the particle speed is $v \equiv c$. 
Due to this, the intensity can be thought of as the energy-track length rate density in the phase space.
Thus, if $\sigma_a$ represents the probability per unit length of a material absorbing a photon, the product $\sigma_a I$ represents the rate energy is deposited by photons at each position, direction, frequency, and time. 
The power deposited in the material is computed with 
	\begin{equation}
		\int_\D \int_{\mathbb{S}^2} \int_0^\infty \sigma_a I \ud \nu \ud\Omega \ud \x \,. 
	\end{equation}
A photon's energy is related to the magnitude of its momentum, $p$, through $E = p c$. A photon traveling in direction $\Omegahat$ then has momentum 
	\begin{equation}
		\vec{p} = \Omegahat \frac{E}{c} = \Omegahat \frac{h \nu}{c} \,. 
	\end{equation}
We then have that $\frac{\sigma_a}{c}\Omegahat I$ represents the rate of momentum deposition in the material due to photon absorption at any point in the phase space. 
The rate of momentum deposited into the material by radiation traveling in all directions and all frequencies is computed with:  
	\begin{equation}
		\frac{1}{c}\int_\D \int_{\mathbb{S}^2} \int_0^\infty \sigma_a\,\Omegahat I \ud \nu \ud \Omega\ud \x \,. 
	\end{equation}
These examples show that the choice of $\psi \equiv vf$ and $I \equiv c h\nu f$ are purely notational conveniences designed to aid in the computation of each field's quantities of interest. 
In particular, the intensity and angular flux are informally related through 
	\begin{equation} \label{trans:psi_I_relation}
		\psi = \frac{I}{h\nu} \quad \mathrm{or} \quad I = E \psi \,. 
	\end{equation}
Here, the informality arises from the angular flux depending on the particle energy and the intensity depending on the particle frequency. The intent of providing Eq.~\ref{trans:psi_I_relation} is to give a simple way to convert between the notations used in nuclear engineering, astrophysics, and \gls{hedp}. 
\textcite[\S 2.1.5]{rgm} provides further commentary on these notational differences. 

\subsection{Moments of the Distribution Function}
Integration over the velocity variable plays an important role in both neutron and photon transport. In the above, we integrated over position-velocity space to determine the reaction rate and the energy and momentum deposition rates. These integrated quantities are called the moments of the distribution function. 
For most purposes, the direction particles are traveling is immaterial for computing reaction and deposition rates. Thus, defining variables that represent integrations of the angular flux over the direction of travel can simplify the above calculations. Let, 
	\begin{equation}
		\phi(\x,E,t) = \int_{\mathbb{S}^2} \psi(\x,\Omegahat,E,t) \ud \Omega \,,  
	\end{equation}
be the \emph{scalar flux}. The reaction rate is then $\int_\D \int_0^\infty \sigma \phi(\x,E,t) \ud E \ud \x$. In this way, the scalar flux represents the track length rate density of neutrons at energy $E$ traveling in any direction.
We will also use the \emph{current} defined as: 
	\begin{equation}
		\vec{J}(\x,E,t) = \int_{\mathbb{S}^2} \Omegahat\,\psi(\x,\Omegahat,E,t) \ud \Omega \,. 
	\end{equation}
Letting $\n$ denote a unit vector normal to a differential area $\ud A$, $\Omegahat\cdot\n\,\psi(\x,\Omegahat,E,t) \ud A$ represents the rate at which particles cross $\ud A$ going in the direction $\Omegahat$ at energy $E$. Thus, the current represents the net number of particles crossing $\ud A$ regardless of $\Omegahat$. We call these terms the zeroth and first moments of the angular flux due to their definition as $\int_{\mathbb{S}^2} \Omegahat^i\,\psi \ud \Omega$ where $i=0$ for the scalar flux and $i=1$ for the current. 

For photons, the astrophysics and \gls{hedp} notations differ in their definitions of the moments. In astrophysics, the zeroth, first, and second moments are defined as averages over solid angle: 
	\begin{subequations}
	\begin{equation}
		J(\x,\nu,t) = \frac{1}{4\pi}\int_{\mathbb{S}^2} I(\x,\Omegahat,\nu,t) \ud \Omega \,, 
	\end{equation}
	\begin{equation}
		\vec{H}(\x,\nu,t) = \frac{1}{4\pi} \int_{\mathbb{S}^2} \Omegahat\, I(\x,\Omegahat,\nu,t) \ud \Omega \,,
	\end{equation}
	\begin{equation}
		\mat{K}(\x,\nu,t) = \frac{1}{4\pi} \int_{\mathbb{S}^2} \Omegahat\otimes \Omegahat\, I(\x,\Omegahat,\nu,t) \ud \Omega \,. 
	\end{equation}
	\end{subequations}
Alternatively, the \gls{hedp} community often uses 
	\begin{subequations}
	\begin{equation} \label{tran:energy_density}
		E(\x,\nu,t) = \frac{1}{c} \int_{\mathbb{S}^2} I(\x,\Omegahat,\nu,t) \ud \Omega \,, 
	\end{equation}
	\begin{equation}
		\vec{F}(\x,\nu,t) = \int_{\mathbb{S}^2} \Omegahat\, I(\x,\Omegahat,\nu,t) \ud \Omega \,,
	\end{equation}
	\begin{equation}
		\P(\x,\nu,t) = \frac{1}{c} \int_{\mathbb{S}^2} \Omegahat\otimes\Omegahat\, I(\x,\Omegahat,\nu,t) \ud \Omega \,. 
	\end{equation}
	\end{subequations}
These moments are referred to as the energy density, flux, and pressure, respectively. Note that the flux does not have the $1/c$ factor that the energy density and pressure have. The astrophysical notation of $J$, $\vec{H}$, and $\mat{K}$ simplifies the radiation transport equation by removing factors of $4\pi$ and $c$ at the expense of introducing those factors into the radiation-hydrodynamics equations \cite{castor2004radiation}. We elect to use the \gls{hedp} notation of $E$, $\mat{F}$, and $\P$ when discussing photon transport and the nuclear engineering notation of $\phi$ and $\vec{J}$ when discussing neutron transport. 

\section{The Equations of Thermal Radiative Transfer}
% simplifying assumptions for rad hydro to TRT 
For thermal photons we consider the linear Boltzmann transport equation given by Eq.~\ref{trans:linboltz} cast in terms of the intensity. Multiplying by $c h\nu $, Eq.~\ref{trans:linboltz} becomes 
	\begin{equation}
		\pderiv{I}{t} + (\Omegahat c) \cdot \nabla I = \paren{\pderiv{I}{t}}_\text{collision} + \paren{\pderiv{I}{t}}_\text{source} \,, 
	\end{equation}
where we have used the photon velocity $\v = \Omegahat c$ and that $I = ch\nu f$. Dividing by $c$ gives the standard form of the transport equation 
	\begin{equation}
		\frac{1}{c}\pderiv{I}{t} + \Omegahat\cdot \nabla I = \frac{1}{c} \paren{\pderiv{I}{t}}_\text{collision} + \frac{1}{c}\paren{\pderiv{I}{t}}_\text{source} \,. 
	\end{equation}
We model scattering and absorption events with the background media by setting: 
	\begin{equation}
		\frac{1}{c}\paren{\pderiv{I}{t}}_\text{collision} = \int_{\mathbb{S}^2} \int_0^\infty \sigma_s(\Omegahat'\rightarrow \Omegahat, \nu' \rightarrow \nu) I(\x,\Omegahat',\nu',t) \ud \nu' \ud \Omega' - \sigma_t I(\x,\Omegahat,\nu,t) \,,
	\end{equation}
where $\sigma_s(\Omegahat'\rightarrow\Omegahat,\nu'\rightarrow \nu)$ is the differential scattering opacity. 
We define 
	\begin{equation}
		\sigma_s(\nu) = \int_{\mathbb{S}^2} \int_0^\infty \sigma_s(\Omegahat'\rightarrow\Omegahat,\nu'\rightarrow\nu) \ud \nu'\ud\Omega'	
	\end{equation}
as the scattering opacity. The total opacity is given by $\sigma_t(\nu) = \sigma_a(\nu) + \sigma_s(\nu)$ where $\sigma_a(\nu)$ is the absorption opacity. In general, all of these opacities also depend on the material's temperature. The source term is used to model the emission of thermal photons by the material. This is modeled by Planck's black body source: 
	\begin{equation}
		\frac{1}{c}\paren{\pderiv{I}{t}}_\text{source} =\frac{\sigma_a(\nu) B(\nu,T)}{4\pi} \,,
	\end{equation}
where
	\begin{equation}
		B(\nu,T) = \frac{2h\nu^3}{c^2}\frac{1}{e^{h\nu/kT}-1}
	\end{equation}
is the Planck emission function with $k$ Boltzmann's constant and $T$ the temperature of the material. Note that the above source is only valid under the assumption of local thermodynamic equilibrium. 
Thus, the Boltzmann transport equation for thermal photons is given by 
	\begin{equation} \label{trans:trans_photon}
		\frac{1}{c}\pderiv{I}{t} + \Omegahat\cdot\nabla I + \sigma_t I = \int_{\mathbb{S}^2}\int_0^\infty \sigma_s(\Omegahat'\rightarrow\Omegahat,\nu'\rightarrow \nu) I(\cdot,\Omegahat',\nu',\cdot) \ud \nu'\ud\Omega' + \frac{\sigma_a B}{4\pi} \,. 
	\end{equation}
Converting the boundary condition in Eq.~\ref{trans:gen_inflow} to apply to the intensity by multiplying by $ch\nu$ and swapping the velocity for the direction of flight yields: 
	\begin{equation}
		I(\x,\Omegahat,\nu,t) = \bar{I}(\x,\Omegahat,\nu,t) \,, \quad \x \in \partial \D \ \mathrm{and} \ \Omegahat\cdot\n < 0 \,, 
	\end{equation}
where $\bar{I}$ is the inflow boundary condition for the intensity. The initial condition supplies $I(\x,\Omegahat,\nu,0)$. 

We now wish to derive an equation for the evolution of the material temperature. We model the material gaining energy by absorbing radiation and losing energy by emitting thermal radiation. Here, we are interested in the energy exchanged between matter and radiation traveling in all directions and at any frequency. 
The energy absorption rate is 
	\begin{equation}
		c\int_0^\infty \sigma_a(\nu) E(\x,\nu,t) \ud \nu \,. 
	\end{equation}
The emission rate is given by the Planck source, $B(\nu,T)$.
The balance of energy between radiation and matter is then written: 
	\begin{equation} \label{trans:meb}
	\begin{aligned}
		C_v(T) \pderiv{T}{t} &= \int_{\mathbb{S}^2} \int_0^\infty \sigma_a(\nu)\!\paren{I(\x,\Omegahat,\nu,t) - \frac{B(\nu,T)}{4\pi}} \ud \nu \ud \Omega \\
		&= \int_0^\infty \sigma_a(\nu) (c E(\x,\nu,t) - B(\nu,T)) \ud \nu \,, 
	\end{aligned}
	\end{equation}
where $C_v(T)$ is the material's heat capacity. 
An initial temperature field must be provided. Boundary conditions for the temperature field are not required since the energy balance equation does not have spatial derivatives. 
The coupled equations given by Eqs.~\ref{trans:trans_photon} and \ref{trans:meb} are called the equations of thermal radiative transfer. This system is nonlinear due to the Planck emission term. 

\section{Steady-State, Linear Transport: A Proxy for TRT}
In this section, the model problem of the steady-state, frequency-independent, linear Boltzmann equation with isotropic scattering is derived from the TRT equations. 

\subsection{Isotropic Scattering}
First, we assume that scattering is isotropic. In other words, any outgoing direction of flight is equally likely for a photon leaving a scattering event. The differential scattering opacity then simplifies to 
	\begin{equation}
		\sigma_s(\Omegahat'\rightarrow\Omegahat,\nu'\rightarrow\nu) \rightarrow \frac{1}{4\pi}\sigma_s(\nu'\rightarrow \nu) \,. 
	\end{equation}
Due to this, the scattering source becomes: 
	\begin{equation}
		\int_{\mathbb{S}^2} \int_0^\infty \sigma_s(\Omegahat'\rightarrow\Omegahat,\nu'\rightarrow \nu) I(\cdot,\Omegahat',\nu',\cdot) \ud \nu'\ud\Omega' \rightarrow \frac{c}{4\pi} \int_0^\infty \sigma_s(\nu'\rightarrow\nu) E(\x,\nu',t) \ud \nu' \,, 
	\end{equation}
where the definition of the frequency-dependent energy density given in Eq.~\ref{tran:energy_density} was used. Isotropic scattering is rarely the correct model for scattering. However, it shares many of the same implementational aspects of more complicated scattering models that have been expanded in spherical harmonics. 

\subsection{Gray Transport}
The gray TRT system is derived by defining suitable frequency-averaged opacities and integrating the transport equation over all angles. The gray opacities are defined as weighted averages of the form: 
	\begin{equation}
		\ubar{\sigma}_x = \frac{\int_0^\infty \sigma_x\,w(\x,\Omegahat,\nu,t) \ud \nu}{\int w(\x,\Omegahat,\nu,t) \ud \nu} \,, 
	\end{equation}
where $x$ is the total, scattering, and absorption events and $w$ is a weight function that approximates the true frequency-dependent intensity. In practice, $w$ could be an approximation to the intensity computed from a larger algorithm with multiple levels in frequency, an analytical approximation to the intensity (e.g.~Rosseland opacities) \cite{pomraning2005equations}, or through another numerical approximation such as with a stochastic method. 

The gray emission rate is the average of the Planck emission source over all directions and frequencies: 
	\begin{equation}
		\ubar{B}(T) = \int_{\mathbb{S}^2} \int_0^\infty B(\nu,T) \ud \nu \ud \Omega = a c T^4 \,,
	\end{equation}
where $a = 8\pi^5 k^4/15h^3 c^3$. 
Letting $\ubar{I}(\x,\Omegahat,t) = \int_0^\infty I(\x,\Omegahat,\nu,t)$ be the frequency-averaged intensity, the gray TRT system is: 
	\begin{subequations}
	\begin{equation}
		\frac{1}{c}\pderiv{\ubar{I}}{t} + \Omegahat\cdot\nabla\ubar{I} + \ubar{\sigma}_t \ubar{I} = \frac{c \ubar{\sigma}_s}{4\pi} E(\x,t) + \frac{\ubar{\sigma}_a c T^4}{4\pi} \,,
	\end{equation}
	\begin{equation}
		\ubar{I}(\x,\Omegahat,t) = \ubar{\bar{I}}(\x,\Omegahat,t) \,, \quad \x\in\partial\D \ \mathrm{and} \ \Omegahat\cdot\n < 0 \,,
	\end{equation}
	\begin{equation}
		C_v(T) \pderiv{T}{t} = c\ubar{\sigma}_a (E(\x,t) - a T^4) \,, 
	\end{equation}
	\end{subequations}
where $\ubar{\bar{I}}(\x,\Omegahat,t) = \int_0^\infty \bar{I}(\x,\Omegahat,\nu,t)\ud\nu$ is the gray inflow boundary function.
The gray TRT equations are often used to iteratively accelerate the convergence of frequency-dependent simulations. In this way, algorithms for the frequency-dependent TRT system must also be effective for the gray TRT system. The gray TRT system then serves as an important first step in the development of frequency-dependent TRT methods. 

\subsection{Steady State}
Time derivatives are neglected under the assumption of steady state. The gray TRT system with isotropic scattering simplifies to 
	\begin{subequations}
	\begin{equation}
		\Omegahat\cdot\nabla\ubar{I} + \ubar{\sigma}_t \ubar{I} = \frac{c \ubar{\sigma}_s}{4\pi} E(\x,t) + \frac{\ubar{\sigma}_a a c T^4}{4\pi} \,,
	\end{equation}
	\begin{equation} \label{trans:graybc}
		\ubar{I}(\x,\Omegahat,t) = \ubar{\bar{I}}(\x,\Omegahat,t) \,, \quad \x\in\partial\D \ \mathrm{and} \ \Omegahat\cdot\n < 0 \,,
	\end{equation}
	\begin{equation}
		0 = c\ubar{\sigma}_a (E(\x,t) - a T^4) \,. 
	\end{equation}
	\end{subequations}
In this case, the material temperature is determined by the radiation field such that: 
	\begin{equation}
		T^4 = \frac{E}{a}  \,. 
	\end{equation}
We are then left with: 
	\begin{equation} \label{trans:trt_simp}
		\Omegahat\cdot\nabla\ubar{I} + \ubar{\sigma}_t \ubar{I} = \frac{c \ubar{\sigma}_s + c\ubar{\sigma}_a}{4\pi} E(\x,t) \,, 
	\end{equation}
along with the gray boundary condition in Eq.~\ref{trans:graybc}. 
This equation models the radiation field at long time scales where the radiation and material have achieved equilibrium. When implicit time integration schemes are applied to the time-dependent TRT system, a steady-state system analogous to Eq.~\ref{trans:trt_simp} with additional sources must be solved at each time step. Thus, the steady-state, frequency-independent TRT system with isotropic scattering represents the core kernel of the full TRT system. 

\subsection{Definition of the Model Problem}
We now define the transport problem that serves as the model for the development of the moment methods presented in this dissertation. The properties of the steady-state, gray TRT system with isotropic scattering can be emulated with the following neutron transport equation: 
	\begin{subequations} \label{trans:model}
	\begin{equation} 
		\Omegahat\cdot\nabla\psi + \sigma_t \psi = \frac{\sigma_s}{4\pi} \int \psi \ud \Omega' + q \,,
	\end{equation}
	\begin{equation}
		\psi(\x,\Omegahat) = \bar{\psi}(\x,\Omegahat) \,, \quad \x \in \partial\D\ \mathrm{and} \ \Omegahat\cdot\n < 0 \,,
	\end{equation}
	\end{subequations}
where $\psi$ represents the energy and time-independent angular flux, $\sigma_t$ and $\sigma_s$ energy-independent cross sections, $q$ a fixed-source of particles, and $\bar{\psi}$ the inflow boundary function for the angular flux. In this document, we solve Eq.~\ref{trans:model} and use the nuclear engineering notation for the moments. That is, we use 
	\begin{equation}
		\phi(\x) = \int_{\mathbb{S}^2} \psi(\x,\Omegahat) \ud \Omega \,,
	\end{equation}
	\begin{equation}
		\vec{J}(\x) = \int_{\mathbb{S}^2} \Omegahat\,\psi(\x,\Omegahat) \ud \Omega \,,
	\end{equation}
	\begin{equation}
		\P(\x) = \int_{\mathbb{S}^2} \Omegahat\otimes\Omegahat\,\psi(\x,\Omegahat)\ud\Omega \,. 
	\end{equation}

\section{The Radiation Diffusion Approximation}
Here, we apply a simplification to the angular dependence in our model transport problem to derive the radiation diffusion approximation. We wish to solve 
	\begin{subequations} \label{trans:model2}
	\begin{equation}
		\Omegahat\cdot\nabla\psi + \sigma_t \psi = \frac{\sigma_s}{4\pi} \int \psi \ud \Omega' + \frac{Q}{4\pi} \,,
	\end{equation}
	\begin{equation}
		\psi(\x,\Omegahat) = \bar{\psi}(\x,\Omegahat)\,, \quad \x \in \partial\D \ \mathrm{and} \ \Omegahat\cdot\n < 0 \,,
	\end{equation}
	\end{subequations}
under the assumption that the angular flux is linearly anisotropic in angle. In other words, the angular flux can be written: 
	\begin{equation}
		\psi(\x,\Omegahat) = \frac{1}{4\pi}(\phi(\x) + 3\Omegahat\cdot\vec{J}(\x) \,. 		
	\end{equation}
We will use the following angular identities: 
	\begin{subequations}
	\begin{equation}
		\int_{\mathbb{S}^2} \ud \Omega = 4\pi \,, 
	\end{equation}
	\begin{equation}
		\int_{\mathbb{S}^2} \Omegahat\otimes\Omegahat \ud \Omega = \frac{4\pi}{3}\I \,, 
	\end{equation}
	\end{subequations}
and that integrating any function that is odd in angle is zero. 
Integrating Eq.~\ref{trans:model2} over all angles yields the zeroth moment equation: 
	\begin{equation}
		\nabla\cdot\vec{J} + \sigma_t \phi = \sigma_s\phi + Q \,. 
	\end{equation}
Multiplying by $\Omegahat$ and integrating over all angles yields the first moment equation: 
	\begin{equation}
		\int_{\mathbb{S}^2}\Omegahat\otimes\Omegahat \cdot \nabla \psi \ud \Omega + \sigma_t \vec{J} = 0 \,, 
	\end{equation}
where the source $Q$ has been assumed to be isotropic. We now use that $\psi$ is linearly anisotropic to simplify the first term: 
	\begin{equation}
	\begin{aligned}
		\int_{\mathbb{S}^2}\Omegahat\otimes\Omegahat \cdot \nabla \psi \ud \Omega &= \nabla \cdot \int_{\mathbb{S}^2}\Omegahat\otimes\Omegahat\,\psi \ud \Omega \\ 
		&= \nabla\cdot \int_{\mathbb{S}^2}\Omegahat\otimes\Omegahat \frac{1}{4\pi}\paren{\phi + 3\Omegahat\cdot\vec{J}} \ud \Omega \\
		&= \nabla\cdot \frac{\phi}{3}\I \\
		&= \frac{1}{3}\nabla\phi \,. 
	\end{aligned}
	\end{equation}
All together, the radiation diffusion system is 
	\begin{subequations}
	\begin{equation}
		\nabla\cdot\vec{J} + \sigma_a \phi = Q \,, 
	\end{equation}
	\begin{equation}
		\frac{1}{3}\nabla\phi + \sigma_t \vec{J} = 0 \,, 
	\end{equation}
	\end{subequations}
where $\sigma_a = \sigma_t - \sigma_s$ was used. By eliminating the current, the second-order form of radiation diffusion is: 
	\begin{equation}
		-\nabla\cdot \frac{1}{3\sigma_t} \nabla\phi + \sigma_a \phi = Q \,. 
	\end{equation}
On the boundary domain, the current is: 
	\begin{equation}
		\vec{J}\cdot\n = \int\Omegahat\cdot\n\,\psi \ud \Omega = \int_{\Omegahat\cdot\n>0} \Omegahat\cdot\n\,\psi \ud \Omega + \int_{\Omegahat\cdot\n<0} \Omegahat\cdot\n\,\bar{\psi} \ud \Omega \,. 
	\end{equation}
That is, the inflow boundary function is used for direction entering the domain while the angular flux is used to compute the outflow from the domain. We now use the linearly anisotropic assumption to simplify the outflow term: 
	\begin{equation}
	\begin{aligned}
		\int_{\Omegahat\cdot\n>0} \Omegahat\cdot\n\,\psi\ud\Omega &= \int_{\Omegahat\cdot\n>0} \Omegahat\cdot\n\,\frac{1}{4\pi}\!\paren{\phi + 3\Omegahat\cdot\vec{J}} \ud \Omega \\
		&= \frac{\phi}{4\pi}\int_{\Omegahat\cdot\n>0} \Omegahat\cdot\n \ud \Omega + \frac{3}{4\pi}\n \cdot \int_{\Omegahat\cdot\n>0} \Omegahat\otimes\Omegahat \ud \Omega \cdot \vec{J} \\
		&= \frac{\phi}{4} + \frac{\vec{J}\cdot\n}{2} \,. 
	\end{aligned}
	\end{equation}
Here, we have used the following half-range angular identities: 
	\begin{equation}
		\int_{\Omegahat\cdot\n>0} \Omegahat\otimes\Omegahat \ud \Omega = \frac{2\pi}{3}\I \,,
	\end{equation}
	\begin{equation}
		\int_{\Omegahat\cdot\n>0} \Omegahat\cdot\n \ud \Omega = \pi \,. 
	\end{equation}
Defining 
	\begin{equation}
		\Jin = \int_{\Omegahat\cdot\n<0} \Omegahat\cdot\n\,\bar{\psi} \ud \Omega 
	\end{equation}
as the inflow partial current, the inflow boundary conditions for the diffusion approximation are: 
	\begin{equation} \label{trans:marshak}
		\frac{\vec{J}\cdot\n}{2} + \frac{\phi}{4} = \Jin \,. 
	\end{equation}
This boundary condition is referred to as a Marshak boundary condition. 
\end{document}