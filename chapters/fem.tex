% !TEX root = ../doc.tex
\documentclass[../doc.tex]{subfiles}

\begin{document}
\chapter{Finite Element Preliminaries}

\section{Local Polynomial Spaces}
Both the mesh and the solution are represented with a polynomial space defined locally on each element. We define 
	\begin{equation}
		\Pcal{k} = \{x^i\}_{i=0}^k = \{ 1, x, x^2, \ldots, x^k \} 
	\end{equation}
as the space of univariate polynomials of degree less than or equal to $k$. The tensor product polynomials are defined as 
	\begin{equation}
		\Qcal{m,n} = \{ p(x) q(y) : p \in \Pcal{m}\,, q \in \Pcal{n} \} 
	\end{equation}
in two dimensions and 
	\begin{equation}
		\Qcal{\ell,m,n} = \{ p(x) q(y) r(z) : p \in \Pcal{\ell}\,, q \in \Pcal{m}\,, r \in \Pcal{n}\} 
	\end{equation}
in three dimensions. 
The tensor production polynomial space of equal degree in each variable is denoted by 
	\begin{equation}
		\Qcal{p} = \begin{cases}
			\Qcal{p,p} \,, & \dim = 2 \\ 
			\Qcal{p,p,p} \,, & \dim = 3 
		\end{cases} \,. 
	\end{equation}
Nodal bases for the space $\Pcal{p}$ are constructed using Lagrange interpolating polynomials. We consider interpolation through the Gauss-Lobatto and Gauss-Legendre points. Let $\{\xi_i\}$ represent the $p+1$ one-dimensional Gauss-Lobatto or Gauss-Legendre points in the interval $[0,1]$. Let $\ell_i$ denote the Lagrange interpolating polynomial satisfying $\ell_i(\xi_j) = \delta_{ij}$ where $\delta_{ij}$ is the Kronecker delta. The set of functions $\{\ell_i\}$ form a basis for $\Pcal{p}$. The basis functions for $\Pcal{1}$, $\Pcal{2}$, and $\Pcal{3}$ through the Gauss-Lobatto and Gauss-Legendre points are shown in Fig.~\ref{fem:shape1d} and \ref{fem:shape_leg1d}, respectively. Nodal bases for the spaces $\Qcal{m,n}$ and $\Qcal{\ell,m,n}$ are formed through tensor products of the corresponding one-dimensional nodal basis. 
% --- shape functions in 1D (Lobatto) --- 
\begin{figure}
\centering
\begin{subfigure}{.30\textwidth}
	\centering
	\includegraphics[width=\textwidth]{figs/shape1.pdf}
	\caption{}
\end{subfigure}
\begin{subfigure}{.30\textwidth}
	\centering
	\includegraphics[width=\textwidth]{figs/shape2.pdf}
	\caption{}
\end{subfigure}
\begin{subfigure}{.30\textwidth}
	\centering
	\includegraphics[width=\textwidth]{figs/shape3.pdf}
	\caption{}
\end{subfigure}
\caption{Plots of the one-dimensional shape functions through the Gauss-Lobatto nodes for (a) linear, (b) quadratic, and (c) cubic polynomial orders.}
\label{fem:shape1d}
\end{figure}

% --- shape functions in 1D (Legendre) --- 
\begin{figure}
\centering
\begin{subfigure}{.30\textwidth}
	\centering
	\includegraphics[width=\textwidth]{figs/shape_leg1.pdf}
	\caption{}
\end{subfigure}
\begin{subfigure}{.30\textwidth}
	\centering
	\includegraphics[width=\textwidth]{figs/shape_leg2.pdf}
	\caption{}
\end{subfigure}
\begin{subfigure}{.30\textwidth}
	\centering
	\includegraphics[width=\textwidth]{figs/shape_leg3.pdf}
	\caption{}
\end{subfigure}
\caption{Plots of the one-dimensional shape functions through the Gauss-Legendre nodes for (a) linear, (b) quadratic, and (c) cubic polynomial orders.}
\label{fem:shape_leg1d}
\end{figure}

Interpolation through the Gauss-Lobatto and Gauss-Legendre points both have the required properties to be accurate in the limit as $p\rightarrow \infty$. Thus, the choice of interpolating points is typically dictated by other aspects of the overall algorithm. Note that the Gauss-Lobatto points include the interval end points $0$ and $1$ while the Gauss-Legendre points do not. The bases resulting from Lagrange interpolation through the Gauss-Lobatto and Gauss-Legendre points are referred to as closed and open, respectively, due to this. The Gauss-Legendre basis has the beneficial property of diagonal mass matrices on affine meshes while the closed Gauss-Lobatto basis typically leads to sparser globally coupled systems since closed bases couple fewer degrees of freedom on interior faces. 

\section{Description of the Mesh}
Let $\D \subset \R^{\dim}$ with $\dim = 2,3$ be the domain of the problem. Consider the tesselation 
	\begin{equation}
		\D = \bigcup_{K \in \meshT} K 
	\end{equation}
with $K$ an element in the mesh $\meshT$. Each element $K$ is obtained as $K = \T(\hat{K})$ where $\T \in [\Qcal{m}]^{\dim}$ is an invertible, polynomial mapping and $\hat{K} = [0,1]^{\dim}$ is the reference element. The mapping $\T$ is derived from a set of global control points and an element-local nodal basis. Figure \ref{fem:quadmesh} shows an example mesh where the control points labeled 2, 7, and 12 are shared so that the mesh coordinates are continuous across the interface between the two elements. Let $\{\ell_i\}$ denote the nodal basis functions for the space $\Qcal{m}$. On each element, the mapping is then 
	\begin{equation}
		\x(\vec{\xi}) = \T(\vec{\xi}) = \sum_i \x_i \ell_i(\vec{\xi}) 
	\end{equation}
where $\x \in K$, $\vec{\xi}\in \hat{K}$, and the $\x_i$ are the control points corresponding to element $K$. Figure \ref{fem:eltrans} depicts the mesh transformation used for the left element of Fig.~\ref{fem:quadmesh}. 
% --- element transformation --- 
\begin{figure}
\centering
\begin{subfigure}{.4\textwidth}
	\centering
	\includegraphics[height=1.75in]{figs/quad_mesh.pdf}
	\caption{}
	\label{fem:quadmesh}
\end{subfigure}
\begin{subfigure}{.59\textwidth}
	\centering
	\includegraphics[height=1.75in]{figs/eltrans.pdf}
	\caption{}
	\label{fem:eltrans}
\end{subfigure}
\caption{Depictions of (a) the mesh control points in a quadratic quadrilateral mesh and (b) the reference transformation used to describe the left element of (a).}
\end{figure}

% The mesh transformations $\T$ are used to facilitate numerical integration on arbitrary elements. 
Letting $\vec{\xi} = \vector{\xi & \eta} \in \hat{K}$ denote the reference coordinates and $\x = \vector{x & y} \in \D$ the physical coordinates such that $\x(\vec{\xi}) = \T(\vec{\xi})$, the Jacobian of the transformation is 
	\begin{equation}
		\F = \pderiv{\x}{\vec{\xi}} = \begin{bmatrix} 
			\pderiv{x}{\xi} & \pderiv{x}{\eta} \\ 
			\pderiv{y}{\xi} & \pderiv{y}{\eta} 
		\end{bmatrix} \,, 
	\end{equation}
with $J = |\F|$ its determinant. The partial derivatives of the mesh transformation are computed by taking derivatives of the nodal basis functions. In other words, 
	\begin{equation}
		\F = \sum_i \x_{i} \otimes \hnabla \ell_i = \sum_{i} \begin{bmatrix} 
			x_{i} \pderiv{\ell_i}{\xi} & x_{i} \pderiv{\ell_i}{\eta} \\
			y_{i} \pderiv{\ell_i}{\xi} & y_{i} \pderiv{\ell_i}{\eta} 
		\end{bmatrix} \,,
	\end{equation}
where $\x_{i} = \vector{x_{i} & y_{i}}$ and $\hnabla$ denotes the gradient with respect to $\vec{\xi}$. 

A mesh transformation is called affine when it can be written as
	\begin{equation}
		\T = \mat{A}\vec{\xi} + b 
	\end{equation}
where $\mat{A}\in\R^{\dim\times\dim}$ and $b\in\R^{\dim}$ are constant with respect to $\vec{\xi}$. In such case, the Jacobian matrix is $\mat{F} = \mat{A}$ and the Hessian of the transformation, defined as $\frac{\partial^2 \x}{\partial^2 \vec{\xi}}$, is identically zero. Quadrilateral elements obtained by scaling, stretching along the $\xi$ or $\eta$ axes, or rotating the reference element are all affine while general quadrilateral elements, such as trapezoidal elements, and curved elements are not affine. 

\section{Integration Transformations}
% --- piola transform --- 
\begin{figure}
	\centering
	\includegraphics[width=.85\textwidth]{figs/piola.pdf}
	\caption{A depiction of the tangent and cotangent bases at the point $\vec{\xi} = (0,0)$ under a non-affine mesh transformation. }
	\label{mms:piola}
\end{figure}

\section{Finite Element Spaces} \label{fem_sec:fes}
% --- RT quiver diagram --- 
\begin{figure}
	\centering
	\includegraphics[width=.5\textwidth]{figs/quiver.pdf}
	\caption{}
	\label{mms:quiver}
\end{figure}

% --- DG FES --- 
\begin{figure}
\centering
\includegraphics[width=.3\textwidth]{figs/dgfes.pdf}
\caption{A depiction of the distribution of degrees of freedom in the linear DG space. The Legendre nodes are used to illustrate that degrees of freedom are not shared between elements. }
\label{fem:dgfes}
\end{figure}

% --- CFEM FES ---
\begin{figure}
\centering
\includegraphics[width=.3\textwidth]{figs/h1fes.pdf}
\caption{A depiction of the distribution of degrees of freedom for the quadratic continuous finite element space. Continuity of members of the finite element space is enforced by sharing degrees of freedom across neighboring elements.}
\label{fem:h1fes}
\end{figure}

% --- RT local polynomial space --- 
\begin{figure}
\centering
\begin{subfigure}{.25\textwidth}
	\centering
	\includegraphics[width=\textwidth]{figs/rt0.pdf}
	\caption{}
\end{subfigure}\qquad
\begin{subfigure}{.25\textwidth}
	\centering
	\includegraphics[width=\textwidth]{figs/rt1.pdf}
	\caption{}
\end{subfigure}\qquad
\begin{subfigure}{.25\textwidth}
	\centering
	\includegraphics[width=\textwidth]{figs/rt2.pdf}
	\caption{}
\end{subfigure}
\caption{The interpolating points used for the nodal basis of the space $\Q_{p+1,p}\times \Q_{p,p+1}$ for (a) $p=0$, (b) $p=1$, and (c) $p=2$. Gauss-Legendre points are used in the tangential direction and Gauss-Lobatto in the normal direction for each component of the vector. Circles denote the degrees of freedom associated with the $\xi$ component and squares the $\eta$ component. }
\label{fem:rt_local_poly}
\end{figure}

% --- RT FES --- 
\begin{figure}
\centering 
\includegraphics[width=.3\textwidth]{figs/rtfes.pdf}
\caption{The distribution of degrees of freedom corresponding to the first degree Raviart Thomas space. Continuity of the normal component is enforced by sharing the degrees of freedom corresponding to the normal component along the interior face between neighboring elements. The circles and squares denote degrees of freedom in the $x$ and $y$ directions, respectively. }
\label{fem:rtfes}
\end{figure}

% --- RT trace space --- 
\begin{figure}
\centering
\includegraphics[width=.3\textwidth]{figs/ifes0.pdf}
\caption{The distribution of degrees of freedom corresponding to $\Lambda_1$, the space defined as the normal trace of the first degree Raviart Thomas space, on a $3\times 3$ mesh. }
\label{fem:ifes}
\end{figure}

\section{Mathematical Notation}
We define $\Gamma$ as the set of unique faces in the mesh with $\Gamma_0 = \Gamma\setminus \partial\D$ the set of interior faces and $\Gamma_b = \Gamma \cap \partial\D$ the set of boundary faces so that $\Gamma = \Gamma_0 \cup \Gamma_b$. We denote the outward unit normal to element $K$ as $\n_K$. On an interior face $\mathcal{F} \in \Gamma_0$ between elements $K_1$ and $K_2$, we use the convention that $\n$ is the unit vector perpendicular to the shared face $K_1 \cap K_2$ pointing from $K_1$ to $K_2$. On such an interior face, the jump, $\jump{\cdot}$, and average, $\avg{\cdot}$, are defined as 
	\begin{equation} \label{eq:jump_avg}
		\jump{u} = u_1 - u_2 \,, \quad \avg{u} = \frac{1}{2}(u_1 + u_2) \,, \quad \mathrm{on} \ \mathcal{F} \in \Gamma_0 \,, 
	\end{equation}
where $u_i = u|_{K_i}$ with analogous definitions for vectors. Note that a continuous function $u$ satisfies $\jump{u} = 0$ on each interior face. 
On boundary faces, the jump and average are set to 
	\begin{equation} \label{eq:jump_avg_bdr}
		\jump{u} = u \,, \quad \avg{u} = u \,, \quad \mathrm{on} \ \mathcal{F} \in \Gamma_b \,,
	\end{equation}
and likewise for vector-valued functions on the boundary. 

Finally, we define the ``broken'' gradient, denoted by $\nablah$, obtained by applying the gradient locally on each element. That is, 
	\begin{equation} \label{eq:broken_grad}
		(\nablah u)|_{K} = \nabla(u|_K) \,, \quad \forall K \in \meshT \,. 
	\end{equation}
This distinction is important for the piecewise polynomial spaces discussed in Section \ref{fem_sec:fes}. 
% --- jumps and averages --- 
\begin{figure}
\centering
\includegraphics[width=.65\textwidth]{figs/jump_avg.pdf}
\caption{A depiction of a discontinuous, piecewise quadratic solution across two quadrilateral elements. The normal vector, $\hat{n}$, is defined as pointing from $K_1$ to $K_2$ along the face between $K_1$ and $K_2$.}
\label{fem:jump_avg}
\end{figure}

\section{Galerkin's Method}

\section{Finite Element Assembly}
% --- fe support diagram --- 
\begin{figure}
\centering
\includegraphics[width=.65\textwidth]{figs/fe_support.pdf}
\caption{A depiction of a mesh of elements and a single basis function. Since the basis functions in the finite element method have local support, each basis function is non-zero only on the elements neighboring the shared node. }
\label{fem:fe_support}
\end{figure}

\section{Iterative Solution Methods for Linear Systems}
\end{document}