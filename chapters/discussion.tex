%!TEX root = ../doc.tex
\documentclass[../doc.tex]{subfiles}

\begin{document}
\chapter{Additional Results and Discussion}
\section{Using Outer Iteration as Initial Guess for Inner Solver}
% --- comparison of convergence --- 
\begin{figure}
\centering 
\begin{subfigure}{.32\textwidth}
	\centering
	\includegraphics[width=\textwidth]{figs/disc/iguess.pdf}
	\caption{}
\end{subfigure}
\begin{subfigure}{.32\textwidth}
	\centering
	\includegraphics[width=\textwidth]{figs/disc/iguess_1.pdf}
	\caption{}
\end{subfigure}
\begin{subfigure}{.32\textwidth}
	\centering
	\includegraphics[width=\textwidth]{figs/disc/iguess_2.pdf}
	\caption{}
\end{subfigure}
\caption{A comparison of the number of inner iterations to convergence at each outer iteration when the inner solver used the previous outer iteration as the initial guess and when an initial guess of zero was used for each iteration. The most refined mesh for polynomials orders (a) 1, (b), 2, and (c) 3 are shown. Using the previous outer iteration as an initial guess reduces the total number of inner iterations required to solve the problem.}
\label{disc:iguess}
\end{figure}

% --- outer and inner on crooked pipe hp --- 
\begin{table}
\centering
\caption{}
\label{}
\input{figs/cp_iguess}
\end{table}

\section{Acceleration of Inexact Sweeps on the Triple Point Mesh}

% --- multiple sweeps on triple point mesh --- 
\begin{table}
\centering
\caption{}
\label{disc:mult_sweeps}
\input{figs/disc/eps3}
\end{table}

% --- anderson acceleration on triple point mesh --- 
\begin{table}
\centering
\caption{}
\label{disc:anderson}
\input{figs/disc/anderson}
\end{table}

\section{Comparison of Methods}

\section{Recommendations for Implementation in Production}

\end{document}