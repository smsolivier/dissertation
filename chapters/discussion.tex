%!TEX root = ../doc.tex
\documentclass[../doc.tex]{subfiles}

\begin{document}
\chapter{Additional Results and Discussion} \label{chap:disc}
\section{Previous Outer Iteration as Initial Guess for Inner Solver}
Here, we discuss the use of the previous fixed-point iteration's solution as the initial guess for the inner linear solve. As the outer iteration converges, it will provide a better and better initial guess for the initial guess. We compare this initial guess to a simple initial of zero for each inner solve. This comparison is performed using the interior penalty VEF method applied to the crooked pipe problem discussed in Section \ref{dgvef_sec:cp}. The inner solver was \gls{bicg} with a tolerance of $10^{-8}$. The outer solver used fixed-point iteration with two Anderson vectors. The outer tolerance was $10^{-6}$. 

Figure \ref{disc:iguess} shows the number of inner iterations required for convergence at each outer iteration for each of the initial guess strategies. These results were generated for each polynomial order using the crooked pipe mesh with 7168 elements. Aside from two cases, the non-zero initial guess required the same or fewer iterations to converge as the solve that used an initial guess of zero. In addition, as the outer solve progresses the number of inner iterations to convergence decreases. The cumulative number of inner iterations as the outer iteration progresses is shown in Fig.~\ref{disc:ciguess}. Using the previous outer iteration as an initial guess led to a reduction in the total number of inner iterations required to solve the problem of 25\%, 30\%, and 32\%  for $p=1$, $p=2$, and $p=3$, respectively, when compared to using a zero initial guess for each inner solve. 
% --- comparison of convergence --- 
\begin{figure}
\centering 
\begin{subfigure}{.32\textwidth}
	\centering
	\includegraphics[width=\textwidth]{figs/disc/iguess.pdf}
	\caption{}
\end{subfigure}
\begin{subfigure}{.32\textwidth}
	\centering
	\includegraphics[width=\textwidth]{figs/disc/iguess_1.pdf}
	\caption{}
\end{subfigure}
\begin{subfigure}{.32\textwidth}
	\centering
	\includegraphics[width=\textwidth]{figs/disc/iguess_2.pdf}
	\caption{}
\end{subfigure}
\caption{A comparison of the number of inner iterations to convergence at each outer iteration when the inner solver used the previous outer iteration as the initial guess and when an initial guess of zero was used for each iteration. The most refined mesh for polynomials orders (a) 1, (b), 2, and (c) 3 are shown. Using the previous outer iteration as an initial guess reduces the total number of inner iterations required to solve the problem.}
\label{disc:iguess}
\end{figure}

% --- cumulative version --- 
\begin{figure}
\centering 
\begin{subfigure}{.32\textwidth}
	\centering
	\includegraphics[width=\textwidth]{figs/disc/ciguess.pdf}
	\caption{}
\end{subfigure}
\begin{subfigure}{.32\textwidth}
	\centering
	\includegraphics[width=\textwidth]{figs/disc/ciguess_1.pdf}
	\caption{}
\end{subfigure}
\begin{subfigure}{.32\textwidth}
	\centering
	\includegraphics[width=\textwidth]{figs/disc/ciguess_2.pdf}
	\caption{}
\end{subfigure}
\caption{A comparison of the cumulative number of inner iterations to convergence as the outer iteration progresses when the inner solver used the previous outer iteration as the initial guess and when an initial guess of zero was used for each iteration. The most refined mesh for polynomials orders (a) 1, (b), 2, and (c) 3 are shown. Using the previous outer iteration as the initial guess reduced the total number of inner iterations to solve the problem by 53, 82, and 118, respectively. This represents a reduction in inner iterations of 25\%, 30\%, and 31\%, respectively.}
\label{disc:ciguess}
\end{figure}

The crooked pipe $hp$ scaling is repeated in Table \ref{disc:outin_cp}. Here, it is seen that the initial guess scheme for the inner iteration does not affect the convergence of the outer iteration. The maximum number of inner iterations across all the outer iterations were within two iterations of each other for both initial guess schemes. However, the minimum number of iterations was typically 30\%-50\% lower when the previous outer iteration's solution was used as an initial guess. This behavior led to fewer total number of inner iterations required to solve the problem.   

% --- outer and inner on crooked pipe hp --- 
\begin{table}
\centering
\caption{The number of Anderson-accelerated fixed-point iterations and the maximum, minimum, and total number of inner iterations performed across all outer iterations for the IP VEF method on the crooked pipe problem refined in $h$ and $p$. An Anderson space of size two is used. The effect of using the previous outer iteration's solution as an initial guess for the inner solver is compared to using an initial guess of zero at each inner iteration. }
\label{disc:outin_cp}
\input{figs/cp_iguess}
\end{table}

\section{Acceleration of Inexact Sweeps on the Triple Point Mesh}
On a high-order mesh with reentrant faces, the transport equation is only approximately inverted at each outer iteration (see the discussion in Section \ref{sn_sec:hosweep}). This means the VEF data are computed from an angular flux possessing errors due to lagging the inflow data on reentrant faces. While VEF converged robustly in the thick diffusion limit on a mesh with reentrant faces, degraded performance for optically thin problems was observed as compared to an equivalent mesh with reentrant faces. Here, we demonstrate two schemes to produce more robust behavior for optically thin problems on curved meshes. The thick diffusion limit problem scales the material data according to 
	\begin{equation}
		\sigma_t = \frac{1}{\epsilon} \,, \quad \sigma_s = \frac{1}{\epsilon} - \epsilon \,, \quad q = \epsilon \,. 
	\end{equation}
The interior penalty VEF method with $p=2$ is used. 

First, we show the effect of using multiple partial transport inversions per outer iteration. That is, at each outer iteration, the approximate inversion of the streaming and collision operator is iterated in order to produce an angular flux that is closer to the angular flux computed with a direct method. Table \ref{disc:mult_sweeps} shows the number of outer iterations required to solve the problem to a tolerance of $10^{-6}$ when one, two, and three partial inversions of the transport problem are performed as $\epsilon \rightarrow 0$. For each value of $\epsilon$, performing more partial sweeps per iteration reduces the total number of iterations to convergence. In fact, for three sweeps, the iterative performance is within two iterations of an equivalent problem on an orthogonal mesh (e.g.~Table \ref{dgvef:tdl}). However, the iterative performance is not improved to the point that fewer total sweeps are performed. While the three sweep option requires the fewest outer iterations, it requires the most sweeps. This suggests that improving robustness to reentrant faces through increasing the number of sweeps per outer iteration comes at a significant cost. A compromise between these two ideas would be to solve the streaming and collision operators to a certain tolerance. This would provide more robustness while potentially reducing the total number of sweeps. 
% --- multiple sweeps on triple point mesh --- 
\begin{table}
\centering
\caption{The number of fixed-point iterations to convergence on the triple point mesh in the thick diffusion limit. On the triple point mesh, the transport equation is not inverted exactly at each iteration due to the presence of re-entrant faces. The performance of the IP VEF method with $p=2$ is compared when 1, 2, and 3 partial inversions of the transport equation are performed at each fixed-point iteration. More transport inversions leads to faster convergence but not to the point that fewer total inversions are performed. }
\label{disc:mult_sweeps}
\input{figs/disc/eps3}
\end{table}

Next, we discuss application of Anderson acceleration to accelerate the outer iteration on a high-order mesh. We compare building the Anderson space from previous iterates of the scalar flux only and from the scalar and angular flux. These are referred to as the ``low memory'' and ''augmented'' options. Note that we store the entire angular flux in the augmented space for implementational ease only. It is possible to store only the degrees of freedom corresponding to reentrant faces. This would be a nonlinear analog of the ideas used for Krylov-accelerated source iteration \cite{doi:10.13182/NSE02-14}. Fixed-point iteration is compared to the two Anderson acceleration schemes in the thick diffusion limit on the triple point mesh in Table \ref{disc:anderson}. The low memory option takes the same or more iterations as fixed-point iteration. This suggests that Anderson applied to the scalar flux only cannot accelerate inexact sweeps due to reentrant faces. The augmented Anderson space resulted in identical convergence for $\epsilon = 10^{-2}$, $10^{-3}$, and $10^{-4}$. For $\epsilon=10^{-1}$, the augmented Anderson scheme converged three iterations faster than fixed-point iteration. This suggests that Anderson acceleration can increase iterative efficiency on thin problems but only when angular flux information is included in the Anderson space. 

% --- anderson acceleration on triple point mesh --- 
\begin{table}
\centering
\caption{The number of Anderson-accelerated fixed-point iterations to solve the thick diffusion limit problem on the triple point mesh. Fixed-point iteration is compared to Anderson-accelerated fixed-point iteration with an Anderson space of five scalar flux solution vectors (Low Memory) and five scalar and angular flux solution vectors (Augmented). The slowdown of the Low Memory option indicates Anderson cannot accelerate the slowdown from inexact sweeps when the angular flux is not included in the Anderson space. }
\label{disc:anderson}
\input{figs/disc/anderson}
\end{table}

\section{Comparison of Methods}
Here, we attempt to provide a coherent, unified discussion of the results presented in Chapters \ref{chap:dgvef}, \ref{chap:rtvef}, and \ref{chap:smm}. Since the mixed finite element method with continuous finite elements for each component of the current could not be efficiently solved, we do not consider it in the discussions presented here. 

\subsection{Solution Quality on the Triple Point Mesh}
Here, we compare the sensitivity to mesh distortion of the VEF methods presented in Chapters \ref{chap:dgvef} and \ref{chap:rtvef}. The thick diffusion limit problem such that 
	\begin{equation}
		\sigma_t = \frac{1}{\epsilon} \,, \quad \sigma_s = \frac{1}{\epsilon} - \epsilon \,, \quad q = \epsilon \,. 
	\end{equation}
with $\epsilon = 10^{-2}$ is used to test solution quality on the triple point mesh. This problem should have a monotonically increasing solution with smooth contours. Deviations from this behavior are due to mesh imprinting from the poor approximation properties of severely distorted elements. 
The scalar flux solutions for the \gls{ip}, \gls{mdldg}, \gls{cg}, and \gls{hrt} VEF methods with $p=2$ are compared in Fig.~\ref{disc:3psol}. Observe that all methods show non-monotonic behavior suggesting all of the methods are sensitive to mesh imprinting. It is interesting to note which methods exhibit continuous contour lines for the solution. The IP method, a DG method that is stabilized through the use of a penalty term, appears to have very similar solution quality to the CG method which solves for the scalar flux in a continuous finite element space. This indicates the effect of the penalty parameter regularizing toward continuous solutions. By contrast, the MDLDG method, a DG method where a penalty term is not used, has discontinuous contours. The HRT method appears to be the most sensitive to mesh distortion as the ``swirl'' in the center of the domain appears the most pronounced. This may be due to the first-order form's weaker solution requirements. In addition, the gradient of the Piola transform required for the Raviart Thomas methods may allow mesh distortion to affect the solution more. This comparison may indicate that the IP and CG methods are more numerically diffusive than the MDLDG and HRT methods.  
% --- solution quality on triple point mesh --- 
\begin{figure}
\centering
\foreach \f in {ip.png, mdldg.png, cg.png, hrt.png}{
	\begin{subfigure}{.49\textwidth}
		\centering
		\includegraphics[width=\textwidth]{data/img/3psol/\f}
		\caption{}
	\end{subfigure}
}
\caption{Plots of the scalar flux from the thick diffusion limit problem with $\epsilon = 10^{-2}$ on the triple point mesh for the (a) interior penalty, (b) MDLDG, (c) CG, and (d) HRT VEF methods. The solution should be smoothly varying and obey a maximum principle. Deviations from this behavior are due to mesh imprinting. These plots show that the IP and CG methods have nearly identical solution quality and in particular have continuous contour lines. The MDLDG and HRT methods produced solutions with discontinuous contour lines.}
\label{disc:3psol}
\end{figure}

\subsection{Thick Diffusion Limit}
Table \ref{disc:all_eps_orthog} shows the number of fixed-point iterations to converge the VEF and SMM algorithms in the thick diffusion limit. This problem is characterized by 
	\begin{equation}
		\sigma_t = \frac{1}{\epsilon} \,, \quad \sigma_s = \frac{1}{\epsilon} - \epsilon \,, \quad q = \epsilon \,. 
	\end{equation}
with the thick diffusion limit corresponding to $\epsilon \rightarrow 0$. A coarse $8\times 8$ mesh is used with $S_4$ angular quadrature. Observe that all the VEF methods converge equivalently for each value of $\epsilon$ tested. The SMM algorithms converged within one iteration of each other. SMM converged 1-3 iterations slower than VEF. On this idealized problem where negativities are unlikely to occur all methods performed very similarly. 
% --- all VEF and SMM on orthogonal mesh TDL ---
\begin{table}
\centering
\caption{The number of iterations until convergence in the thick diffusion limit for all the methods presented in this dissertation for $p=2$. An $8\times 8$ orthogonal mesh is used with $S_4$ angular quadrature. The fixed-point tolerance was $10^{-6}$.}
\label{disc:all_eps_orthog}
\input{figs/disc/eps_all}
\end{table}

The thick diffusion limit problem was repeated on the triple point mesh in Table \ref{disc:all_eps_3p}. Here, the presence of reentrant faces prevents the ability to solve the transport equation exactly with an element-by-element sweep. The transport equation is instead solved with an approximate sweep that lags the incoming angular flux data corresponding to reentrant faces. This means the transport equation is not exactly inverted at each fixed-point iteration, slowing convergence of the VEF and SMM algorithms as compared to the orthogonal mesh problem. In addition, the severely distorted elements have poor approximation ability. This problem represents a difficult stress test. These results indicate that all VEF and SMM methods are robust to reentrant faces. However, it is seen that the MDLDG, RT, and HRT VEF methods are slower to converge compared to the IP, BR2, and CG VEF methods. This is likely due to an increase in negativities for the methods that are less numerically diffusive. A similar behavior is seen for the RT and HRT SMMs as compared to the IP and CG SMMs. Again, VEF generally converges a few iterations faster than SMM. 
% --- all VEF and SMM on triple point TDL --- 
\begin{table}
\centering
\caption{The number of iterations until convergence in the thick diffusion limit for all the methods presented in this dissertation for $p=2$. The triple point mesh is used with $S_4$ angular quadrature. The fixed-point tolerance was $10^{-6}$.}
\label{disc:all_eps_3p}
\input{figs/disc/eps_all_1}
\end{table}

\subsection{Crooked Pipe}
Next, we compare performance on the multi-material crooked pipe problem. This problem is defined in Sections \ref{dgvef_sec:cp}, \ref{rtvef_sec:cp}, and \ref{smm_sec:cp}. The problem consists of two materials with an 1000x difference in total interaction cross section and corresponds to the first Newton iteration of a time-dependent \gls{trt} algorithm. A comparison of all the VEF and SMM methods presented here is provided in Table \ref{disc:all_cp_outer}. Anderson-accelerated fixed-point iteration is used with a tolerance of $10^{-6}$. Two Anderson vectors are used. The zero and scale negative flux fixup was used for the DG VEF methods. The mixed finite element VEF methods and all the SMMs used the quadratic programming negative flux fixup. We again see that the less numerically diffusive methods require 1-3 more iterations to converge than the numerically diffusive methods. This is true for both the VEF and SMM methods. All methods show a slight increase in iterations as the polynomial order is refined. This is likely due to the increased reliance on the negative flux fixup for higher polynomial orders. 
Even on this difficult problem, SMM converged only slightly slower than the corresponding VEF method. 
% --- outer iterations for all VEF and SMM on crooked pipe --- 
\begin{table}
\centering
\caption{The number of Anderson-accelerated fixed-point iterations until convergence on the crooked pipe problem for all the methods presented in this dissertation. An Anderson space of size two was used. The iterative tolerance was $10^{-6}$. }
\label{disc:all_cp_outer}
\input{figs/cp_outerall}
\end{table}

Table \ref{disc:all_cp_inner} shows the average number of inner preconditioned linear solver iterations until a convergence of $10^{-8}$ across all the outer iterations on the crooked pipe problem. Note that the VEF methods used \gls{bicg} while the SMMs used iterative schemes that leverage the symmetry of the SMM diffusion operator. In particular, the IP VEF method used \gls{bicg} whereas the IP SMM used the conjugate gradient method. In this case, each iteration for the VEF method is twice as expensive as the SMM iteration since \gls{bicg} performs both a matrix-vector product and a transpose matrix-vector product where the conjugate gradient method only requires one matrix-vector product per iteration. Here, we attempt only to show that all of the methods can be scalably solved with respect to both the mesh size and the polynomial order. A comparison of performance would require timing data. 
% --- avg. inner iterations for all VEF and SMM on crooked pipe ---
\begin{table}
\centering
\caption{The average number of inner iterations until convergence across all the outer iterations on the crooked pipe problem. }
\label{disc:all_cp_inner}
\input{figs/cp_innerall}
\end{table}

\subsection{Weak Scaling}
Table \ref{disc:allweak} summarizes the weak scaling performance of all the methods presented here. The data is taken from Sections \ref{dgvef_sec:weak}, \ref{rtvef_sec:weak}, and \ref{smm_sec:weak}. The first iteration of the crooked pipe problem is used to demonstrate the convergence of the inner solve. The tolerance for the preconditioned linear solvers was $10^{-8}$. The VEF methods used \gls{bicg} while the IP, CG, and HRT SMMs used conjugate gradient. The RT SMM used \gls{bicg}. Both IP methods used the uniform subspace correction preconditioner with one Jacobi iteration and one AMG V-cycle per iteration. The CG and HRT VEF methods and SMMs used one AMG V-cycle as a preconditioner. Both RT methods used a lower block triangular preconditioner that used one iteration of Gauss-Seidel to approximate the inverse of the total interaction mass matrix and one AMG V-cycle to approximate the inverse of the lumped Schur complement. 
% --- weak scaling on crooked pipe for all methods --- 
\begin{table}
\centering
\caption{A weak scaling study of the inner solve on the first iteration of the crooked pipe for all the VEF methods and SMMs presented in this dissertation. The inner solver tolerance was $10^{-8}$.}
\label{disc:allweak}
\input{figs/allweak}
\end{table}

Observe that all methods can be solved efficiently in parallel on a mesh of over a million elements. For both VEF and SMM, the CG and HRT solvers converged similarly. The RT and IP methods generally required the most iterations. Comparing VEF and to SMM, it is seen that the SMM solvers require more iterations. This is due to the use of conjugate gradient instead of \gls{bicg} for the IP, CG, and HRT methods. For the case of RT VEF and SMM where \gls{bicg} was used for both types, the iteration counts are very close with both methods converging within $\pm 2$ iterations of each other. 

\subsection{A Qualitative Discussion of Cost}
The primary costs in each iteration of a VEF or SMM algorithm are: the transport sweep, computing the VEF/SMM data, forming the VEF/SMM system, and solving the VEF/SMM system. All methods share the costs of the transport sweep and computing the VEF/SMM data. Furthermore, computation of the VEF and SMM data have a similar cost. For example, computing the Eddington tensor requires using angular quadrature to compute the second and zeroth moment of the discrete angular flux and forming their ratio. The SMM correction tensor also computes the second and zeroth moment of the discrete angular flux but instead subtracts them. Thus, the methods are differentiated by the cost of forming and inverting the system. 

\todo{finish}
\end{document}