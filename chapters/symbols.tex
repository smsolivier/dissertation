%!TEX root = ../doc.tex
\documentclass[../doc.tex]{subfiles}

\begin{document}
Below is a list of the symbols and mathematical notation used throughout this document. 

\subsection*{Mathematical Symbols}
\begin{longtable}{p{2cm}p{12cm}}
$\in$ & indicates set membership. $x \in A$ means that $x$ is an element of the set $A$\\
$\subset$ & indicates subset. $A \subset B$ if all elements of $A$ are also members of $B$ \\
$\emptyset$ & the emtpy set defined as the unique set that has no elements \\
$\cap$ & intersection of two sets. $A\cap B$ has only the elements that belong to both $A$ and $B$ \\
$\cup$ & union of two sets. $A\cup B$ has all the elements of $A$ and $B$ \\
$\R$ & the set of real numbers \\
$\dim$ & denotes spatial dimensionality ($1\leq \dim \leq 3$) \\
$\D$ & denotes the computational domain, $\D \subset \R^{\dim}$\\ 
$\partial\D$ & the boundary of the domain, $\D$ \\
$\mathbb{S}^2$ & the unit sphere \\
$\e_i$ & the canonical Cartesian basis \\
$\nabla$ & the spatial gradient $\nabla = \pderiv{}{x} \e_x + \pderiv{}{y}\e_y + \pderiv{}{z}\e_z$ \\
$\nabla\cdot$ & the divergence operator \\
$:$ & scalar contraction of two tensors \\
$\x$ & the spatial variable such that $\x \in \D$ \\
$\ud\x$ & differential volume element such that $\ud\x = \ud x \ud y \ud z$ \\ 
$\ud s$ & differential surface element (e.g.~for integrating over an embedded surface) \\
$t$ & time variable \\
$\mathcal{O}$ & of the order \\
$\equiv$ & defined as \\
\end{longtable}

\subsection*{Finite Element Notation}
\begin{longtable}{p{2cm}p{12cm}}
$\mathcal{P}_k(\hat{K}^1)$ & space of univariate polynomials of degree less than or equal to $k$ \\
$\mathcal{Q}_{m,n}(\hat{K})$ & tensor product polynomial space of $\mathcal{P}_m(\hat{K}^1)$ and $\mathcal{P}_n(\hat{K}^1)$ \\
$\mathcal{Q}_{\ell,m,n}(\hat{K})$ & tensor product polynomial space of $\mathcal{P}_\ell(\hat{K}^1)$, $\mathcal{P}_m(\hat{K}^1)$, and $\mathcal{P}_n(\hat{K}^1)$ \\
$\mathcal{Q}_p(\hat{K})$ & tensor product space of equal degree in each coordinate \\ 
$\ell_i(\vec{\xi})$ & a nodal basis function for a polynomial space defined on $\hat{K}$ \\
$\hat{K}^{\dim}$ & the $\dim$-dimensional reference element, $\hat{K}^{\dim} = [0,1]^{\dim}$ \\
$\vec{\xi}$ & referential coordinate such that $\vec{\xi} \in \hat{K}$ \\
$K$ & a finite element \\
$\T$ & reference to physical space transformation such that $K = \T(\hat{K})$ \\
$\F$ & Jacobian matrix of an element transformation \\
$J$ & Jacobian of an element transformation such that $J = \det(\F)$ \\
$h$ & characteristic mesh element length \\
$\Qbb{p}(K)$ & space of mapped polynomials defined by composing $\Qcal{p}$ with the inverse mesh transformation \\
$\mathbb{D}_k(K)$ & element-local polynomial space for the Raviart Thomas space $\RT_k$ \\
$\n$ & outward unit normal vector \\ 
$\n_K$ & outward normal with respect to element $K$ \\
$\hat{\n}$ & outward unit normal vector to a surface on the reference element $\hat{K}$ \\
$\tang$ & unit vector tangential to a surface \\
$b_i$ & denotes a global basis function for a finite element space \\
$\kappa$ & interior penalty parameter \emph{or} \\ 
& condition number of an algebraic system of equations \\

$\meshT$ & collection of elements $K$ called a tessellation of the domain, $\D$ \\
$\jump{\cdot}$ & the jump of a function across a face between two elements \\
$\avg{\cdot}$ & the average of a function evaluated on each side of a face \\
$\nablah$ & the local gradient defined by applying the gradient locally on each element \\ 
$\hnabla$ & gradient in reference space \\
$\mathcal{F}$ & denotes a face in the mesh \\
$\Gamma$ & Set of unique faces in the mesh \\
$\Gamma_0$ & Set of unique interior faces in the mesh \\
$\Gamma_b$ & Set of unique faces on the boundary of the mesh, $\Gamma = \Gamma_0 \cup \Gamma_b$ \\
$L^2(\D)$ & the space of square-integrable functions \\ 
$H^1(\D)$ & the space of square-integrable functions with square-integrabl gradient \\ 
$H(\div;\D)$ & the space of square-integrable vector-valued functions with square-integrable divergence \\ 
$Y_p$ & the degree-$p$ Discontinuous Galerkin space \\
$X_p$ & the vector-valued degree-$p$ Discontinuous Galerkin space ($X_p = [Y_p]^{\dim}$)\\
$V_p$ & the degree-$p$ continuous finite element space \\ 
$W_p$ & the degree-$p$ vector-valued finite element space where each component of the vector belongs to $V_p$ ($W_p = [V_p]^{\dim}$) \\ 
$\RT_p$ & the order $p$ Raviart Thomas finite element space \\ 
$\Lambda_p$ & the interior trace of the order $p$ Raviart Thomas space, $\RT_p$ \\
\end{longtable}

\subsection*{Radiation Transport}
\begin{longtable}{p{2cm}p{12cm}}
$h$ & Planck's constant \\
$\Omegahat$ & the direction-of-flight variable. $\Omegahat\in \mathbb{S}^2$\\
$\nu$ & photon frequency \\
$E$ & neutron energy \\
$\psi$ & the angular flux \\
$\bar{\psi}$ & the inflow boundary function \\
$J_n^\pm$ & half-range partial currents of the angular flux \\
$\Jin$ & the inflow current computed from the inflow boundary function, $\bar{\psi}$ \\
$\phi$ & the zeroth angular moment of the angular flux \\
$\vec{J}$ & the first angular moment of the angular flux \\
$\P$ & the second angular moment of the angular flux \\
$\E$ & the Eddington tensor defined as the ratio of the second and zeroth angular moments of the angular flux \\
$E_b$ & the Eddington boundary factor used to form boundary conditions for the VEF method \\
$\T$ & SMM correction tensor \\
$\beta$ & the SMM boundary correction factor \\
$Q_0$ & the zeroth moment of the fixed-source, $q$ \\
$\Qone$ & the first moment of the fixed-source, $q$ \\
$\sigma_s$ & the scattering macroscopic cross section \\
$\sigma_a$ & the absorption macroscopic cross section \\
$\sigma_t$ & the total interaction macroscopic cross section such that $\sigma_t = \sigma_s + \sigma_a$ \\ 
\end{longtable}
\end{document}