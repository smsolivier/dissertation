%!TEX root = ../doc.tex
\documentclass[../doc.tex]{subfiles}

\begin{document}
\chapter{Second Moment Methods}

\section{The Continuous Algorithm}
\glspl{smm} can be viewed as using an alternative closure of the second moment of the angular flux. Where \gls{vef} uses $\P = \E \varphi$, \gls{smm} uses 
	\begin{equation}
		\P = \T(\psi) + \frac{1}{3}\I \varphi 
	\end{equation}
where $\T(\psi)$ is a \emph{correction tensor} defined as: 
	\begin{equation}
		\T(\psi) = \int \Omegahat\otimes\Omegahat\, \psi \ud \Omega - \frac{1}{3}\I \varphi \,. 
	\end{equation}
Note that this is simply an algebraic reformulation of the second moment $\P = \int \Omegahat\otimes\Omegahat\, \psi \ud \Omega$ where the isotropic tensor $1/3\I\varphi$ is added and subtracted. 
Like the VEF closure, the SMM closure is trivial in that the solution to the transport equation must already be known in order to define the correction tensor. 

The SMM algorithm forms an efficient iterative scheme by lagging the correction tensor. Using this closure in the zeroth and first moment equations yields: 
	\begin{subequations}
	\begin{equation}
		\nabla\cdot\vec{J} + \sigma_a \varphi = Q_0 \,, 
	\end{equation}
	\begin{equation}
		\nabla\cdot\paren{\T(\psi) + \frac{1}{3}\I \varphi} + \sigma_t \vec{J} = \vec{Q}_1 \,. 
	\end{equation}
	\end{subequations}
The second-order form is found by eliminating the current: 
	\begin{equation}
		-\nabla\cdot\frac{1}{3\sigma_t}\nabla\varphi + \sigma_a \varphi = Q_0 - \nabla\cdot\frac{\Qone}{\sigma_t} + \nabla\cdot\frac{1}{\sigma_t}\nabla\cdot\T(\psi) \,. 
	\end{equation}
SMM solves the coupled system: 
	\begin{subequations}
	\begin{equation}
		\Omegahat\cdot\nabla\psi + \sigma_t \psi = \frac{\sigma_s}{4\pi}\varphi + q \,, 
	\end{equation}
	\begin{equation}
		-\nabla\cdot\frac{1}{3\sigma_t}\nabla\varphi + \sigma_a \varphi = Q_0 - \nabla\cdot\frac{\Qone}{\sigma_t} + \nabla\cdot\frac{1}{\sigma_t}\nabla\cdot\T(\psi) \,. 
	\end{equation}
	\end{subequations}
Let $\mat{L} = \Omegahat\cdot\nabla + \sigma_t$, $\mat{S} = \frac{\sigma_s}{4\pi}$, and 
	\begin{equation}
		\mat{D} = -\nabla\cdot\frac{1}{3\sigma_t}\nabla + \sigma_a
	\end{equation}
be the streaming and collision, scattering, and diffusion operators. By eliminating the angular flux, the SMM algorithm is equivalent to: 
	\begin{equation}
		\mat{D}\varphi = Q_0 - \nabla\cdot\frac{\Qone}{\sigma_t} + \nabla\cdot\frac{1}{\sigma_t}\nabla\cdot\T(\mat{L}^{-1}(\mat{S}\varphi + q)) \,. 
	\end{equation}
The key aspect of SMM is that the left hand side operator $\mat{D}$ is no longer a function of the angular flux and represents a radiation diffusion system. By applying the inverse of the diffusion system to both sides, we arrive at the fixed-point operator 
	\begin{equation}
		\varphi = G(\varphi)
	\end{equation}
where 
	\begin{equation}
		G(\varphi) = \mat{D}^{-1}\bracket{Q_0 - \nabla\cdot\frac{\Qone}{\sigma_t} + \nabla\cdot\frac{1}{\sigma_t}\nabla\cdot\T(\mat{L}^{-1}(\mat{S}\varphi + q))} \,. 
	\end{equation}
Note that for SMM, evaluating $G(\varphi)$ only requires the inversion of the streaming and collision operator and a symmetric positive definite diffusion system. \todo{advantageous because}

\section{Derivation by Linearizing VEF}

\section{A Connection to Newton's Method}

\section{Discrete Second Moment Methods}
\subsection{Interior Penalty and Continuous Finite Element}

\subsection{Mixed Finite Element}

\section{Results}

% --- mms figures --- 
\begin{figure}
\centering
\begin{subfigure}{.4\textwidth}
	\centering
	\includegraphics[width=\textwidth]{figs/smm/mms.pdf}
	\caption{}
\end{subfigure}
\begin{subfigure}{.4\textwidth}
	\centering
	\includegraphics[width=\textwidth]{figs/smm/mms_1.pdf}
	\caption{}
\end{subfigure}
\begin{subfigure}{.4\textwidth}
	\centering
	\includegraphics[width=\textwidth]{figs/smm/mms_2.pdf}
	\caption{}
\end{subfigure}
\end{figure}

% --- MFEM MMS --- 
\begin{table}
\centering
\caption{}
\label{}
\input{figs/smm/mms_diff}
\end{table}

\begin{table}
\centering
\caption{}
\label{}
\input{figs/smm/vmms}
\end{table}

\begin{table}
\centering
\caption{}
\label{}
\input{figs/smm/mms_elev}
\end{table}

% --- thick diffusion limit --- 
\begin{table}
\centering
\caption{}
\label{}
\input{figs/eps_table_smm}
\end{table}

% --- crooked pipe hp scaling --- 
\begin{table}
\centering
\caption{}
\label{}
\begin{adjustbox}{max width = \textwidth}
\input{figs/cp_smm}
\end{adjustbox}
\end{table}

% --- weak scaling --- 
\begin{table}
\centering
\caption{}
\label{}
\input{figs/smm/weak}
\end{table}

\end{document}